%Pretreatment========================================================================================
\documentclass[12pt]{article}
\usepackage{lingmacros}
\usepackage{tree-dvips}
\usepackage{graphicx}
\usepackage{hyperref}
\usepackage{amsmath}
\usepackage{amssymb}
\usepackage{multicol}
\usepackage{geometry}
\usepackage{cite}
\usepackage[amsmath,thmmarks]{ntheorem}
\usepackage{algpseudocode}
\usepackage{algorithm}
\usepackage{listings} 
\usepackage{verbatim}
\usepackage{subfigure}
\usepackage{appendix}  
\usepackage{color}
\usepackage{wrapfig}
\usepackage[dvipsnames]{xcolor}
%\usepackage{subcaption}


\theoremstyle{plain}
\theoremseparator{\hspace{1em}} \theoremnumbering{arabic}
\theoremsymbol{}
\newtheorem{theorem}{\textbf{Theorem}}[section]
\newtheorem{definition}{\textbf{Definition}}[section]
\newtheorem{lemma}{\textbf{Lemma}}[section]
\newtheorem{proof}{\textit{PROOF}}[section]
\newtheorem{example}{\textbf{E x a m p l e}}[section]
\newtheorem{problem}{\textbf{P r o b l e m}}[section]
\newtheorem{solution}{\textit{SOLUTION}}[section]
\newtheorem{discussion}{\textit{D I S C U S S I O N}}[section]
\newtheorem{conclusion}{\textit{\textbf{CONCLUSION}}}[section]

 
\geometry{left=2cm,right=2cm,top=3cm,bottom=2cm}
\title{[Note] Chaos: An Introduction to Dynamical Systems}
\author{}
\date{\today}

\begin{document}
\maketitle

{
\begin{center}
\LARGE \textbf{Problem in discrete-time system}
\end{center}
}
\section{One-Dimension Maps}

\begin{definition}\textbf{n-order differentiable function, Smooth function, Map}
\\\noindent Consider an open set $E$ and $n \in \mathcal N$, called 
$$
C^n(E) = \{f \in C(E) | \forall \alpha \text{ s.t. } |\alpha| \leq n, D^\alpha f \in C(E)\}
$$
is n-order differentiable function set of $E$, where $C(E)$ is continuous function on $E$.
\\\noindent If $f$ on domian $E$ have infinity-order derivative, or $f \in C^\infty(E)$, then called $f$ \textbf{smooth function}.
\\\noindent If the function $f$ have same domain and range, then called $f$ is a \textbf{map}.
\end{definition}

{\color{red} The fucntion in this book will be a smooth function if we not emphasize.}

\begin{definition}\textbf{Orbit, initial value, fixed point}
\\\noindent Consider a map $f: X \rightarrow X$, $x$ is a point in $X$ then 
\\\noindent Called \textbf{orbit} of $x$ is a set of point 
$$
\text{Orbit}(X) = \{x, f(x), f^2(x), \ldots f^n(x), \ldots\}
$$
, where $f^n(x) = f(f(\ldots f(x))) = (f\circ f \circ f \circ \ldots \circ f)(x)$.
\\\noindent The starting point of $x$ for a orbit called the \textbf{initial value}.
\\\noindent If the point $p$ s.t. $f(p) = p$, then called $p$ as fixed point.
\end{definition}

OK, and now we consider two dynamical systems, with a input $x$, the system will always return to $f(x) = 2x$ and $g(x) = 2x(1-x)$. And then the output will become the input value and etc. During this looping, it is simple to find the orbit of a certain initial value.


\begin{table}[H]
\centering  
\caption{Comparison of exponential growth and logistic growth}  
\begin{tabular}{|c||c|c|c|c|c|c|c|c|c|c|}
\hline
f & init & 1 & 2 & 3 & 4 & 5 & 6 & 7 & 8 & 9 \\
\hline
\hline
$f(x)$ & 0.01 & 0.02   & 0.04   & 0.08   & 0.16   & 0.32     & 0.64        & 1.28  & 2.56  & 5.12  \\
\hline
$f(x)$ & 0.01 & 0.0198 & 0.0388 & 0.0746 & 0.138  & 0.268    & 0.362       & 0.462 & 0.497 & 0.499 \\
\hline
$g(x)$ & 0.5  & 0.5    & 0.5    & 0.5    & 0.5    & 0.5      & 0.5         & 0.5   & 0.5   & 0.5   \\
\hline
$g(x)$ & 0.8  & 0.32   & 0.435  & 0.492  & 0.499  & 0.5      & 0.5         & 0.5   & 0.5   & 0.5   \\
\hline
$g(x)$ & 1.2  & -0.48  & -1.42  & -6.87  & -108.4 & -23716.9 & -1125030476 & -inf  & -inf  & -inf  \\
\hline
\end{tabular}  
\end{table}  

We found that in the model of $f$, the result is growth as exponential function and we called that exponential growth. Also, when intial value $x \in [0, 1]$, with iteration, the result have limition of 0.5 and we called these model as logistic growth.

In this section, we will mainly focus on these kind of dynamical system, obviously, the iteration processing of model are discrete, we also called these dynamical system models as \textbf{maps}.






\subsection{Cobweb plot, stability}
To analysis a maps, the basic method is based on cobweb plot. Fig \ref{cobweb-plow-1} showed a method to analysis a dynamical system with a certain iteration principle. 
\begin{figure}[H]
\begin{center}
\includegraphics[width=0.6\textwidth]{figure/section1/cobweb-plot.png} \\
\caption{An example of cobweb plot and basic principle}\label{cobweb-plow-1}
\end{center}
\end{figure}

In every iteration, the independent and dependent variable exchanged their location and we can found a group of $\{x_1, y_2, x_3, y_4, \ldots\}$ as orbit of initial value. Or, with the symmetric line $y = x$, it is simple to symmetric all black line to purple line and we can build a cobweb plot with origin image and $y = x$ to find a group of $\{x_1, x_2, \ldots\}$ as orbit from the initial value.

Before we discuss the different of fixed point, it is necessary to review some basic definitions.

\begin{definition}\textbf{$\varepsilon$ Neighbourhood}
\\\noindent In a metric space $X$, an $\varepsilon$ neighbourhood $N_\varepsilon(p)$ of point $p$ is defined 
$$
N_\varepsilon(p) = \{x \in X | d(x, p) < \varepsilon\}
$$
where $d(x, p)$ is the distance bewteen point $p$ and $x$. Also, in a $R^1$ space, the $\varepsilon$ Neighbourhood is give by
$$
N_\varepsilon(p) = \{x \in R | |x-p| < \varepsilon\}
$$
\end{definition}


\newpage

\begin{figure}[H]
\begin{minipage}[c][0.6\width]{
   0.5\textwidth}
   \centering
   \includegraphics[width=1\textwidth]{figure/section1/cb01.png}
\end{minipage}
\begin{minipage}[c][0.6\width]{
   0.5\textwidth}
   \centering
   \includegraphics[width=1\textwidth]{figure/section1/cb02.png}
\end{minipage}
\begin{minipage}[c][0.6\width]{
   0.5\textwidth}
   \centering
   \includegraphics[width=1\textwidth]{figure/section1/cb03.png}
\end{minipage}
\begin{minipage}[c][0.6\width]{
   0.5\textwidth}
   \centering
   \includegraphics[width=1\textwidth]{figure/section1/cb04.png} \\
\end{minipage}
\caption{Cobweb plot in different initial value}\label{logistic-cobweb-plot}
\end{figure}


It is simple to find that for all initial value $x \in (0, 1)$, with iteration, the output have limitation in 0.5. On the other hand, to solve the equation $x = 2x(1-x)$ we found $x_1 = 0, x_2 = 1, x_3 = 0.5$ as three fixed point. So we have two kinds of fixed point, the one is limitation point and the other is not.


\begin{definition}\textbf{Sink, Source (Attracting and Repelling Fixed Point)}
\\\noindent Consider a map $f: R \rightarrow R$ and point $p$ s.t. $f(p) = p$, then
\\\noindent If for evert points sufficiently to $p$ are attracted to $p$, then called $p$ as \textbf{sink}, or \textbf{attracting fixed point}. Or
$$
\text{For an } \varepsilon > 0, \forall x \in N_\varepsilon(p), \lim_{k \rightarrow \infty}f^k(x) = p \text{ then called } p \text{ as \textbf{sink}}.
$$
If for every points sufficiently to $p$ are repelled to $p$, then called $p$ as \textbf{source}, or \textbf{repelling fixed point}.
$$
\text{For an } \varepsilon > 0, \forall x \in N_\varepsilon(p), x \neq p, \lim_{k \rightarrow \infty}f^k(x) \notin N_\varepsilon(x) \text{ then called } p \text{ as \textbf{source}}.
$$
\end{definition}



\newpage

\begin{theorem} \label{sink-source-point}Let $f$ is a map on $R$, assume $p$ is a fixed point of $f$, then
\\\noindent [i] If $|f'(p)| < 1$, then $p$ is a sink;
\\\noindent [ii] If $|f'(p)| > 1$, then $p$ is a source.
\end{theorem}


{\color{blue}
\begin{proof} \textbf{[i]} Based on definition of derivative, we have
$$
\lim_{x \rightarrow p} {|f(x) - f(p)| \over |x - p|} = |f'(p)|
$$
Now. let $a \in (\min(|f'(p)|, 1), \max(|f'(p)|, 1))$ (e.g. $a = {1\over 2}(1+ |f'(p)|)$), then
$$
\forall a \in (\min(|f'(p)|, 1), \max(|f'(p)|, 1)), \exists \varepsilon_0 > 0 \text{ s.t. } \forall \varepsilon \in (0, \varepsilon_0], \forall x \in N_\varepsilon(p), {|f(x) - f(p)| \over |x - p|} < a
$$
            \textit{That means, $f(x)$ is closer to $p$ than $x$ (or distant bewteen curve $y = f(x)$ and $y = x$), but at least a factor of $a$ and we have the conclusion
$$
\forall x \in N_\varepsilon(p), f(x) \in N_\varepsilon(p)
$$
            During the iteration processing it is simple to find that all orbit $\{f(x), f^2(x), \ldots f^n(x), \ldots\} \subset N_\varepsilon(p)$, so now we can consider another conclusion in follow.}
\\[2ex]\noindent \textbf{[ii]} We try to prove the inequality $\forall x \in N_\varepsilon(p), |f^k(x) - p|$
\\\noindent \textbf{[ii-1]} Obvious, if $k = 1$, then $|f(x) - p| = |f(x) - f(p)| < a |x-p|$ ($p$ is fixed point so $f(p) = p$)
\\\noindent \textbf{[ii-2]} If $k = 2$, Based on the conclusion in \textbf{[i]}, $x_1 = f(x) \in N_\varepsilon(p)$ and $|f(x_1) - p| < a |x_1 - p| < a^2 |x - p|$
\\\noindent $\ldots$
\\\noindent \textbf{[ii-k+1]} (Assume the inequality is established in $k$), then
$$
|f^{k+1}(x) - p| < a|f^k(x) - p| < a \cdot a^k |x - p| = a^{k+1}|x - p|
$$
            In summary, for all $k \in N$, the inequality is established.
\\[2ex]\noindent \textbf{[iii-1]} Now we consider the equality condition, if $|f'(p)| < 1$ then $a < 1$ and  
$$
\lim_{k \rightarrow \infty}|f^{k}(x) - p| < |x-p|\lim_{k \rightarrow \infty} a^k = 0 \text{ (Because }a \in (0, 1)\text{)} 
$$
            So we have the conclusion, 
$$
\forall x \in N_\varepsilon(p), \lim_{k \rightarrow \infty}f^k(x) = p
$$
            \textbf{[iii-2]} Also, if $|f'(p)| > 1$ then $a^k \rightarrow \infty$, that means, with the iteration, the maps will eventually outside the condition, or the domain interval. $\blacksquare$
\end{proof}
}

* We will discuss what happened while $f'(p) = 1$ laterly.

** Obviously, this theorem expressed a kind of convergence, as the speed of the convergence is based on the $a$ in exponent function, we called this convergence as \textbf{Exponential Convergence}.

Now we consider another map as example.


\newpage
\begin{example} Solved the fixed point of $\varphi(x) = (3x -x^2)/2$, find every sink and source point with Theo. \ref{sink-source-point}
\end{example}


\begin{figure}[H]
\begin{center}
\includegraphics[width=0.5\textwidth]{figure/section1/cobweb-plot-2.png} \\
%\caption{}\label{cobweb-plow-2}
\end{center}
\end{figure}

{\color{blue}
\begin{solution}
It is simple to find the fixed point with $x = (3x - x^3) / 2$ and $x_1 = 1, x_2 = 0, x_3 = -1$. Based on the image, we can found that $1$ and $-1$ are sink and $0$ is source. On the other hand
$$
\varphi'(x) = {3\over 2} (1 - x^2), \varphi'(-1) = 0 < 1, \varphi'(0) = {3 \over 2} > 1, \varphi'(1) = 0 < 1
$$ 
and we proved the conclusion we found on figure before. $\blacksquare$
\end{solution}
}

Another way to confirm a point is sink or source is based on the formula identity and algebra. For instance, we consider the distance bewteen $g(x) = 2x(1-x)$ and fixed point $1/2$, then 
$$
|g(x) - 1/2| = |2x(1-x) - 1/2| = 2|x - 1/2||x - 1/2|
$$

and $\forall x \in (0, 1), |x - 1/2| < 1 \Rightarrow |g(x) - 1/2| < 1$, that means the distance bewteen $g(x)$ and $p$ is decreasing during time iteration and we can confirm that $1/2$ is a sink point rather than source point.

Next, we will focus on a logistic model with different parameter.








\subsection{Periodic points, family of logistic maps}
\begin{example} Find the fixed point of $g(x) = 3.3x(1-x), x \in [0, 1]$.
\end{example}

{\color{blue}
\begin{solution}
It is simple to find the fixed point with $x = 3.3x(1-x)$ and $x_1 = 0, x_2 = 23/33, x_3 = 1$. Obviously, both 0 and 1 are source. And
$$
g'(x) = 3.3 - 6.6x, |g'(23/33)| = 1.3 > 1
$$ 
So all these three fixed point are source, and it is simple to find the conclusion with cobweb plot. $\blacksquare$
\end{solution}
}

\newpage
\begin{figure}[H]
\begin{center}
\includegraphics[width=0.6\textwidth]{figure/section1/periodic-point.png} \\
\caption{An example of periodic point}\label{periodic-point}
\end{center}
\end{figure}


Hold on a second, something strange! Even we cannot find a sink fixed point, all of initial value are sank into a group of points!

\begin{definition}\textbf{Period-k point, Period-k orbit}
\\\noindent Let $f$ be a map on $R$, and $p$ is a point in domain, if $f^k(p) = p$, and $k$ is the smallest such positive integer, then called $p$ as \textbf{periodic point of period $k$}, or \textbf{period-k point};
\\\noindent Called orbit with initial point $p$ as \textbf{periodic orbit of period k}, or \textbf{period-k orbit};
\end{definition}

\begin{definition}\textbf{Sink and Source in Period point}
\\\noindent Let $f$ be a map and $p$ is a period-k point
\\\noindent If $p$ is a sink, then called this period-k orbit as periodic sink;
\\\noindent If $p$ is a source, then called this period-k orbit as periodic source.
\end{definition}

Obviously, based on the chain rule, we have $(fg)'(x) = f'(g(x))g'(x)$, let $f = g, x = p_1$, then 
$$
g^2(p_1) = g'(g(p_1))g'(p_1) = g'(p_2)g'(p_1)
$$
Summary this formula, we have 
\begin{theorem} \label{chain-rule-period-orbit}For every map $f$ and period-k orbit $\{p_1, p_2, \ldots p_k\}$, 
$$
(f^k)'(p_1) = (f^k)'(p_2) = \ldots = (f^k)'(p_k) = \prod_{i = 1}^{k}f'(p_i)
$$
\end{theorem}

{\color{blue}
\begin{proof} \textbf{Theo. \ref{chain-rule-period-orbit}} 
$$
(f^k)(p_1) = (f(f^{k-1}))'(p_1) = f'(f^{k-1}(p_1))(f^{k-1})'(p_1) = \ldots = \prod_{i = 1}^{k}f'(p_i) = (f^k)(p_i) (\forall i = 1, 2, \ldots, k)\blacksquare
$$
\end{proof}
}

\newpage
Same as Theo. \ref{sink-source-point}, we have stability test for periodic orbits.
\begin{theorem}\label{stability-test-for-periodic-orbits}\textbf{Stability test for periodic orbits}
\\\noindent Let $f$ is a map and period-k orbit $\{p_1, p_2, \ldots p_k\}$, 
\\\noindent If $|\prod_{i = 1}^{k}f'(p_i)| < 1$ then called this periodic orbit is a sink;
\\\noindent If $|\prod_{i = 1}^{k}f'(p_i)| > 1$ then called this periodic orbit is a source;
\end{theorem}


{\color{blue}
\begin{proof} \textbf{Theo. \ref{sink-source-point}} 
\\\noindent Consider a new map $g(x) = f^k(x)$, where $f$ be a map and $p$ is a period-k point, then $p$ is a fixed point of $g$. Based on Theo. \ref{sink-source-point}, $|g(p)| < 1$ if $p$ is sink and $|g(p)| > 1$ if $p$ is a source. On the other hand, $g(p) = f^k(p) = \prod_{i = 1}^{k}f'(p_i) \blacksquare$   
\end{proof}
}

Now we consider another problem.

\begin{example} Find the fixed point or periodic orbit of $g_{3.5}(x) = 3.5x(1-x), g_{3.86}(x) = 3.86x(1-x), x \in [0, 1]$.
\end{example}


\begin{figure}[H]
\begin{minipage}[c][0.5\width]{
   0.5\textwidth}
   \centering
   \includegraphics[width=0.9\textwidth]{figure/section1/logistic35.png}
\end{minipage}
\begin{minipage}[c][0.5\width]{
   0.5\textwidth}
   \centering
   \includegraphics[width=0.9\textwidth]{figure/section1/logistic386.png} \\
\end{minipage}
\caption{Logistic maps in $a = 3.5$ and $a = 3.86$}\label{logistic-no-periodic}
\end{figure}

We found in $a = 3.5$, even the periodic orbit is difficult to find, the iteration still have a boundary. If we consider every $a \in [1, 4]$, we can plot a figure between parameter $a$ and orbits $x$, and this \textbf{bifurcation diagram} was made by following repearting:
\\\noindent \textbf{[i]} Choose a value $a$, starting with $a = 1$.
\\\noindent \textbf{[ii]} Choose a value $x \in [0, 1]$ randomly.
\\\noindent \textbf{[iii]} Calculate the orbit of x under $g_a(x)$ in a certain iteration times $t_max$.
\\\noindent \textbf{[iv]} Ignore the first $t_0$ iterates and plot the orbit.


\begin{figure}[H]
\begin{center}
\includegraphics[width=0.4\textheight]{figure/section1/logistic-stability.png} \\
\caption{Logistic model stability interval ($a \in [1, 4]$)}\label{Logistic-stability}
\end{center}
\end{figure}




\newpage
\begin{discussion} Now we will discuss the family of logistic maps with Fig. \ref{Logisstic-stability}
\\\noindent \textbf{[i] Periodic-3 window}
\\\noindent We found periodic-1 orbits (or point) and periodic-2 orbits, based on the image above, it seem we also have periodic-3 orbits. And now we focus on the interval of parameter $a$ rather than domain of function, we found there is a interval of $a$ inside the $[3.83, 3.86]$ and we called these kind of interval as ``periodic window''. For instance, next figure showed the periodic-3 window of $a$.

\begin{figure}[H]
\begin{minipage}[c][0.5\width]{
   0.5\textwidth}
   \centering
   \includegraphics[width=0.70\textwidth]{figure/section1/periodic-3-window.png}
\end{minipage}
\begin{minipage}[c][0.5\width]{
   0.5\textwidth}
   \centering
   \includegraphics[width=0.9\textwidth]{figure/section1/logistic384.png} \\
\end{minipage}
\caption{Periodic-3 window and cobweb plot in $a = 3.84$}\label{logistic-cobweb-plot1}
\end{figure}

That's fine, let's check the result by cobweb plot. Ok, hold on a second, something wrong! So we still need more analysis.

Obviously, every periodic-3 orbit of $g$ is a fixed point of $g^3$, so we can also analysis $g^3$ map.\\[3ex]

\begin{figure}[H]
\begin{minipage}[c][0.33\width]{0.33\textwidth}
   \centering
   \includegraphics[width=\textwidth]{figure/section1/g3logistic-origin.png}
\end{minipage}
\begin{minipage}[c][0.33\width]{0.33\textwidth}
   \centering
   \includegraphics[width=\textwidth]{figure/section1/g3logistic384.png}
\end{minipage}
\begin{minipage}[c][0.33\width]{0.33\textwidth}
   \centering
   \includegraphics[width=\textwidth]{figure/section1/g3logistic384-001-detail.png} 
\end{minipage}
\\[3ex]\caption{$g^3$ map and $g^3$ cobweb plot figure}\label{logistic-cobweb-plot1}
\end{figure}

We found different from periodic-2 orbit, the periodic-3 orbit is nearby(rather than equal) the point and it seems we have periodic-3 orbit. Actually, we will explain all periodic-3 will implies a characteristic we called ``chaos''. 






\newpage
  \noindent \textbf{[ii] The Logistic Map $G(x) = 4x(1-x)$}
\\\noindent Now we consider another logistic map where $a \equiv 4$. \\[1ex]
\\\noindent Firstly, why we are interested in $g_4(x)$, consider a quadratic function 
$$
g_a(x) = ax(1-x) = a(-x^2 + x - 1 + 1) = -a(x-{1\over 2})^2 +{a\over 4}
$$
this function have maximum at point $x = 1/2$ and the maximum is $a/4$. As we have the Theo. \ref{sink-source-point}, if we consider the sink point set, it is necessary to satisfy $|g_a(x)| < 1$, or${a \over } 4 < 1 \Rightarrow a < 4$. So at the point $a = 4$, this set is empty and this is a critical state. For every $a_{new} = a - \varepsilon (\varepsilon \rightarrow 0)$, we have the interval of sink. So at this point, some special property has been result and that is why we interested in this map.

We can still find the fixed point of $g_4(x)$ to solve $g_4(x) = x$, and we have $x_{11} = 0, x_{21} = {3/4}$. If we consider periodic-k orbit, for instance, we consider periodic-2 orbit, then we have solve the function $g(g(x)) = x$ as 
$$
g(g(x)) = 4(4x(1-x))[1-4x(1-x)] = x \Rightarrow (4x^2-4x+1)(x-1)x+{x\over 16} = 0 
$$
$$
\Rightarrow (4x-3)(16x^2 - 20x+5)x = 0 \Rightarrow x_{21} = 0, x_{22} = {3\over 4}, x_{23,24} = {5\pm \sqrt{5}\over 8}
$$

Also, it is easy to check the periodic-k orbit in the figure.\\[2ex]

\begin{figure}[H]
\begin{minipage}[c][0.24\width]{0.24\textwidth}
   \centering
   \includegraphics[width=\textwidth]{figure/section1/4-logistic-ite-1.png}
\end{minipage}
\begin{minipage}[c][0.24\width]{0.24\textwidth}
   \centering
   \includegraphics[width=\textwidth]{figure/section1/4-logistic-ite-2.png}
\end{minipage}
\begin{minipage}[c][0.24\width]{0.24\textwidth}
   \centering
   \includegraphics[width=\textwidth]{figure/section1/4-logistic-ite-3.png}
\end{minipage}
\begin{minipage}[c][0.24\width]{0.24\textwidth}
   \centering
   \includegraphics[width=\textwidth]{figure/section1/4-logistic-ite-4.png}
\end{minipage}
\\[3ex]\caption{$g_4^1, g_4^2, g_4^3$ and $g_4^4$ figure}\label{4-logistic-ite}
\end{figure}

\begin{figure}[H]
\begin{center}
\includegraphics[width=0.5\textwidth]{figure/section1/4-logistic-cobweb-plot.png}\\
\caption{$g_4(x)$ cobweb plot(periodic-1,2 orbits)}\label{4-logistic-ite}
\end{center}
\end{figure}

\newpage
We found a conclusion here
\begin{conclusion}
For every periodic-k, the model $g_4^k$ have $2^{k} - 1$ saddle-node bifurcation and $2^k$ fixed point. And these $2^k$ points include every fixed point for model $g_4^{i}, i = 1, 2, \ldots, k-1 \land k \equiv 0 (\text{mod } i)$
\end{conclusion}

The number of orbits of the map for each period can be tabulated in the map's periodic table.



\begin{table}[H]
\centering  
\caption{The periodic table for the logistic4 map}  
\begin{tabular}{|c||c|c|c|c|c|c|c|}
\hline
Period $k$                             & 1 & 2 & 3 & 4  & 5  & 6  & 7   \\
\hline
\hline
Number of fixed points of $g_4^k$      & 2 & 4 & 8 & 16 & 32 & 64 & 128 \\
\hline
Orbits of Period $k$                   & 2 & 1 & 2 & 3  & 4  & 5  & 6   \\
\hline
Fixed points due to lower orbits       & 0 & 2 & 2 & 4  & 2  & 4  & 2   \\
\hline
\hline
1                                      & / &   &   &    &    &    &   \\
2                                      & * & / &   &    &    &    &   \\
3                                      & * &   & / &    &    &    &   \\
4                                      & - & * &   & /  &    &    &   \\
5                                      & * &   &   &    & /  &    &   \\
6                                      & - & * & * &    &    & /  &   \\
7                                      & * &   &   &    &    &    & / \\
\hline
\end{tabular}  
\end{table}
(*: Greatest common divisor group, -: $g^k$ fixed)
\end{discussion}







\subsection{Chaos}
We still focus on $g_4(x)$ map, we try to check the $g_4^2$ fixed point ${5 - \sqrt{5}\over 8}$.\\[2ex]

\begin{figure}[H]
\begin{minipage}[c][0.24\width]{0.24\textwidth}
   \centering
   \includegraphics[width=\textwidth]{figure/section1/4-logistic-stable-30.png}
\end{minipage}
\begin{minipage}[c][0.24\width]{0.24\textwidth}
   \centering
   \includegraphics[width=\textwidth]{figure/section1/4-logistic-stable-50.png}
\end{minipage}
\begin{minipage}[c][0.24\width]{0.24\textwidth}
   \centering
   \includegraphics[width=\textwidth]{figure/section1/4-logistic-stable-100.png}
\end{minipage}
\begin{minipage}[c][0.24\width]{0.24\textwidth}
   \centering
   \includegraphics[width=\textwidth]{figure/section1/4-logistic-stable-500.png}
\end{minipage}
\\[3ex]\caption{$g_4^1, g_4^2, g_4^3$ and $g_4^4$ figure}\label{4-logistic-ite}
\end{figure}

It seems something wrong. Because we proved that ${5 \pm \sqrt{5}\over 8}$ is a periodic orbit during the iteration, but once we growth the iteration times, the results filled all the interval.

So what happened? We try to put all of our data into a same image, and we have
\begin{figure}[H]
\begin{center}
\includegraphics[width=0.8\textwidth]{figure/section1/4-logistic-stable-check.png} \\
\caption{Iteration and ``periodic-2 orbit'' value}\label{periodic-2-check}
\end{center}
\end{figure}

Obviously, in about first 40 times iteration, it was worked for a while, but with the iteration increasing, the error also increaed rapidly. Ok, ok, let's check the data for more details.
\begin{table}[H]
\centering  
\caption{Logistic4 periodic-2 orbit iteration}  
\begin{tabular}{|c||c|c|c|c|}
\hline
1-4   & 0.3454915028125262  & 0.9045084971874737  & 0.3454915028125262  & 0.9045084971874735 \\
\hline
5-8   & 0.34549150281252694 & 0.9045084971874745  & 0.3454915028125237  & 0.9045084971874705 \\
\hline
9-12  & 0.34549150281253665 & 0.9045084971874865  & 0.3454915028124849  & 0.9045084971874225 \\
\hline
13-16 & 0.34549150281269186 & 0.9045084971876783  & 0.3454915028118641  & 0.9045084971866552 \\
\hline
17-20 & 0.3454915028151752  & 0.9045084971907479  & 0.34549150280193086 & 0.9045084971743771 \\
\hline
21-24 & 0.3454915028549078  & 0.9045084972398602  & 0.3454915026430002  & 0.904508496977928  \\
\hline
25-27 & 0.3454915034906304  & 0.9045084980256565  & 0.34549150010010987 & $\ldots$           \\
\hline
\end{tabular}  
\end{table}

We noticed that during the iteration, the values of periodic-2 orbit are actually changed very small. Then we realized that is beacause of ${5 - \sqrt{5}\over 8} \neq 0.3454915028125262$ and this is just a value near the periodic point.(And the computer can only calculate this estimation value rather than real value.) Even this two value are almost nearby, it still have a little difference, and this difference become larger and larger during the iteration.

That is important beacause we found even two value are almost equal, after iterate, this tiny, tiny difference will become a catastrophe and eventually two orbits move apart.

\begin{definition}\textbf{Sensitive dependence on initial conditions, Sensitive point}
\\\noindent Let $f$ is a map on $R$, $x_0$ in domain.
\\\noindent If there is a nonzero distance $d$ s.t. some points arbitrary near $x_0$ are eventually mapped at least $d$ units from the corresponding image of $x_0$, then we called $x_0$ has \textbf{sensitive dependence on initial conditions};
\\\noindent If for this $x_0, \exists \varepsilon > 0$ s.t. $\forall x \in N_\varepsilon^o(x_0) = N_\varepsilon(x_0) \backslash \{x_0\}, \exists K$ s.t. $\forall k > K, ||f^k(x) - f^k(x_0)|| \geq \varepsilon$, then called this point is \textbf{sensitive point}.
\end{definition}
\begin{definition}\textbf{Eventually periodic}
\\\noindent Let $f$ is a map on $R$, $x_0$ in domain. If for some positive integer $N, \forall n > N, f^{n+p}(x) = f^n(x)$, then we called $x$ \textbf{eventually periodic} with period p, where p is the smallest such positive integer.
\end{definition}

Now we consider another model to explain this definition in another way.

\begin{example}Consider a map $f(x) = 3x (\text{mod } 1)$. (e.g. $f(4.33) = 0.33, f(-1.98) = 0.02$.)
\end{example}

\begin{figure}[H]
\begin{center}
\includegraphics[width=0.4\textwidth]{figure/section1/3xmod1.png} \\
\caption{3x mod 1 cobweb plot(initial value: {\color{green}0.25(green)}, {\color{red}0.2501(red)})}\label{3xmod1}
\end{center}
\end{figure}

\newpage
Basically, we have
\begin{theorem} For any map $f$, the source has sensitive dependence on initial conditions.
\end{theorem}
{\color{blue}
\begin{proof} For a certain $\varepsilon$, as $p$ is a source, then $\forall x \in N^o_\varepsilon(p), \lim_{k \rightarrow \infty}f^k(x) \notin N^\varepsilon_d(p) \Rightarrow d(p, x) > \varepsilon$ $\blacksquare$   
\end{proof}
}

Is there any way to investigate this sensitive dependence? Yes, and here we will introduce a method called \textbf{itinerary} of an orbit.

{\color{blue}
\begin{solution}\textbf{Itinerary}
\\\noindent We still consider the $g_4$ model. Assign the symbol \textbf{L} to the left subinterval $[0, 1/2]$ and \textbf{R} to the right subinterval $[1/2, 1]$. Then, for every initival condition $x_0$, we can list the itinerary with \textbf{L} and \textbf{R}. 
\\\noindent For instance, the initial point $x_0 = 1/3$ have the itinerary \textbf{LRLRLRRLLRR}
\begin{table}[H]
\centering  
\caption{Logistic4 $1/3$ itinerary}  
\begin{tabular}{|c||c|c|c|}
\hline
0-2   & 0.333333333333333  (\textbf{L}) & 0.888888888888889  (\textbf{R})  & 0.39506172839506154 (\textbf{L}) \\
\hline
3-5   & 0.9559518366102727 (\textbf{R}) & 0.16843169076687667(\textbf{L})  & 0.5602498252491516 (\textbf{R})  \\
\hline
6-8   & 0.9854798342297868 (\textbf{R}) & 0.05723732222487492 (\textbf{L}) & 0.21584484467760304(\textbf{L})  \\
\hline
9-10  & 0.6770233908148179 (\textbf{R}) & 0.874650876417697   (\textbf{R}) & $\ldots$                         \\
\hline
\end{tabular}  
\end{table}

And we can list all itinerary with different initial value.$\blacksquare$
\begin{table}[H]
\centering  
\caption{Logistic4 itinerary with different initial value}  
\begin{tabular}{|c||l|l|l|l|l|}
\hline
Val  & $1-10$              & $11-20$             & $21-30$             & $31-40$             & $\ldots$ \\
\hline
\hline
0.01 & \textbf{LLLRRLLLLR} & \textbf{RRRLRLRLRR} & \textbf{LLRRRLLRRR} & \textbf{RLRRLRLRLR} & $\ldots$ \\
\hline
0.25 & \textbf{LRRRRRRRRR} & \textbf{RRRRRRRRRR} & \textbf{RRRRRRRRRR} & \textbf{RRRRRRRRRR} & $\ldots$ \\
\hline
1/3  & \textbf{LRLRLRRLLR} & \textbf{RLRLLRRRRL} & \textbf{LLLRRRRLRL} & \textbf{RRRRRRRLRR} & $\ldots$ \\
\hline
0.5  & \textbf{RRLLLLLLLL} & \textbf{LLLLLLLLLL} & \textbf{LLLLLLLLLL} & \textbf{LLLLLLLLLL} & $\ldots$ \\
\hline
1    & \textbf{RLLLLLLLLL} & \textbf{LLLLLLLLLL} & \textbf{LLLLLLLLLL} & \textbf{LLLLLLLLLL} & $\ldots$ \\
\hline
\end{tabular}  
\end{table}

\end{solution}
}




Notice that there are some conclusions.
\begin{conclusion} For every periodic-k point, the itinerary of orbit will repeats \textbf{L} or \textbf{R} infinitely.
\end{conclusion}

\begin{conclusion} For every $k$ iterate, the itinerary have $2^k$ choice and the sum of their lengths is 1(or the length of the interval).
\end{conclusion}

Also, we have a conclusion not very obvious.

\begin{conclusion} Each $2^k$ itinerary is shorter than $\pi / 2^{k+1}$.
\end{conclusion}
We will prove this conclusion in later sections.



\newpage
We can also analysis the problem with \textbf{transition graph}.

\begin{figure}[H]
\begin{center}
\includegraphics[width=0.2\textwidth]{figure/section1/transition-graph-1.png} \\
\caption{Transition graph}\label{transition-graph-1}
\end{center}
\end{figure}

Finally, we focus on the title of this subsection ``chaos'', after these analysis, it is simple to summary the definition of chaos.

\begin{definition}\label{chaos-orbit}\textbf{Chaos}
\\\noindent A chaotic orbit is a bounded, non-periodic orbit that displays sensitive dependence. Chaotic orbits seoarate exponentially fast from their neighbors as the map iterated.
\end{definition}


\begin{theorem} \label{p3-chaos}The existence of periodic-3 orbit alone implies the existence of a large set of sensitive points, or chaotic orbit.
\end{theorem}

We will prove this problem in appendix.

\newpage

%~~~~~~~~~~~~~~~~~~~~~~~~~~~~~~~~~~~~~~~~~~~~~~~~~~~~~~~~~~~~~~~~~~~~~~~~~~~~~~~~~~~~~~~~~~~~~~~~~~~~~













%~~~~~~~~~~~~~~~~~~~~~~~~~~~~~~~~~~~~~~~~~~~~~~~~~~~~~~~~~~~~~~~~~~~~~~~~~~~~~~~~~~~~~~~~~~~~~~~~~~~~~

\section{Two-Dimension and High-Dimension Maps}

In this section, we will mainly discusse a new type of model, called Henon map which formed
$$
f(x, y) = (a - x^2 + by, x) 
$$


A simple way to analysis this problem is analysis all point in the surface if they are convergence or divergence. In figures following, point in black represent initial conditions whose orbits diverge to infinity and the points in white represent initial values whose orbits converge to the period-2 orbit.\\[4ex]
\begin{figure}[H]
\begin{minipage}[c][0.24\width]{0.24\textwidth}
   \centering
   \includegraphics[width=\textwidth]{figure/section2/Henon-orbit-0-0*4.png}
\end{minipage}
\begin{minipage}[c][0.24\width]{0.24\textwidth}
   \centering
   \includegraphics[width=\textwidth]{figure/section2/Henon-orbit-2--0*3.png}
\end{minipage}
\begin{minipage}[c][0.24\width]{0.24\textwidth}
   \centering
   \includegraphics[width=\textwidth]{figure/section2/Henon-orbit-1*4--0*3.png}
\end{minipage}
\begin{minipage}[c][0.24\width]{0.24\textwidth}
   \centering
   \includegraphics[width=\textwidth]{figure/section2/Henon-orbit-1*28--0*3.png}
\end{minipage}
\\[6ex]\caption{Initial condition square}\label{Initial-condition-square}
\end{figure}
(Parameter group $(a, b) = (0, 0.4), (2, -0.3), (1.4, -0.3), (1.28, -0.3))$







\subsection{Analysis of Henon map}
Now we focus on Henon map. Familiar with 1 dim map, it is necessary to define the sink and source as well as saddle.

\begin{definition}\textbf{Neighborhood}
\\\noindent Consider a $R^n$ space, called every point $x = (x_1, x_2, \ldots x_n)$ is a vector of $R^n$ space,
\\\noindent Define the \textbf{Euclidean Length} $|x| = \sqrt{x_1^2 + x_2^2 + \ldots + x_n^2}$, which is equal to norm;
\\\noindent And define the distance between two point $d(x, y) = |x - y|$;
\\\noindent Also, the \textbf{$\mathbf{\varepsilon}$-neighborhood} is 
$$
\forall \varepsilon > 0, \text{the }\varepsilon \text{-neighborhood of point }p, N_\varepsilon(p) \text{ is } \{x\in R^n | |x - p| < \varepsilon\} \text{, also define }N_\varepsilon^o(p) = N_\varepsilon(p)\backslash\{p\}
$$
\end{definition}

\begin{definition}\textbf{Sink and Source in High-dimension Map}
\\\noindent Let $f$ is a map on $R^n$, $p$ is a vector on $R^n$ which is the fixed point and $f(p) = p$ then 
\\\noindent If there is an $\varepsilon > 0$ s.t. $\forall x \in N_\varepsilon(p), \lim_{k \rightarrow \infty}f^k(x) = p$, then $p$ is a sink or attracting fixed point.
\\\noindent If $\forall x \in N_\varepsilon^o(p), \exists K \text{s.t.} \forall k > K, f^k(x) \notin N_\varepsilon(p)$, then called the point $p$ as source.
\end{definition}


We will explain these definitions with an example


\begin{example} \label{Henon-map-0-0*4}Analysis the sink point, source point and saddle of Henon map with parameter $a = 0, b = 0.4$
\end{example}

{\color{blue}
\begin{solution} Obviously, if we consider the funcion $f(x, y) = (-x^2 + 0.4y, x) = (x, y)$, then 
$$
 -0.2x^2 +0.4x = x \Rightarrow x_1 = 0, x_2 = -0.6
$$
So the fixed points are $(0, 0)$ and $(-0.6, -0.6)$. And now we have a new problem: how to confirm a fixed point is sink or source. Even the definition of sink and source are given above, we still need theory like Theo. \ref{sink-source-point}. But here, we can analysis the problem with simulator.$\blacksquare$\\[2ex]

\begin{figure}[H]
\begin{minipage}[c][0.25\width]{0.25\textwidth}
   \centering
   \includegraphics[width=\textwidth]{figure/section2/Henon-0-0*4-sink.png}
\end{minipage}
\begin{minipage}[c][0.45\width]{0.45\textwidth}
   \centering
   \includegraphics[width=\textwidth]{figure/section2/Henon-0-0*4-source.png}
\end{minipage}
\begin{minipage}[c][0.25\width]{0.25\textwidth}
   \centering
   \includegraphics[width=\textwidth]{figure/section2/Henon-0-0*4-saddle.png} 
\end{minipage}
\\[3ex]\caption{Sink, source and saddle in Henon map with $a = 0, b = -0.4$}\label{sink-source-saddle-henon-map}
\end{figure}
(Order of color: {\color{black}Black(neighborhood)}, {\color{red} Red (Iter = 1)}, {\color{green} Green (Iter = 2)}, {\color{blue} Blue (Iter = 3)})

\end{solution}
}



To solve the problem we faced in e.g.\ref{Henon-map-0-0*4}, we will discuss the simple form of the high dimension maps.

\begin{definition}\textbf{High dimension linear map}
\\\noindent A map $A: R^m \rightarrow R^m$ is \textbf{linear} if $\forall a, b \in R, \forall x, y \in R^m$, $f(ax + by) = af(x) + bf(y)$. Equivalently, a lienar map $f(x)$ can be represented as multiplication by an $m \times m$ matrix.
\end{definition}

Now we consider a system s.t. $f(x) = Ax$, if $\lambda$ is eigenvalue and $\mathbf v$ is eigenvector of $A$, based on the definition fo eigenvalue and eigenvector, we have
Let $A$ have eigenvalue $\lambda$, based on the definition of eigenvalue, we have
$$
A\mathbf x = \lambda \mathbf v
$$
Then, for the initial point $\mathbf v$, we have
$$
A(\mathbf v) = A \mathbf v = \lambda \mathbf v
$$
let $\mathbf v_0 = \mathbf v, \mathbf v^n = A^n(\mathbf v)$, then
$$
\mathbf v_1 = A \mathbf v_0 = \lambda\mathbf v_0, \mathbf v_2 = A \mathbf v_1 = \lambda^2\mathbf v_1 \ldots \mathbf v_n = A \mathbf v_{n-1} = \lambda^{n}\mathbf v_0
$$
Futhermore, if we consider a system in random initial value $\mathbf x_0$, still define $\mathbf x_n = f(\mathbf x_{n-1})$, then 
$$
\mathbf x_n = f(\mathbf x_{n-1}) = A\mathbf x_{n-1} = Af(\mathbf x_{n-2}) = \ldots = A^n \mathbf x_0
$$





To analysis this problem, firstly we will review some theorems in algebra.







\begin{discussion}\textbf{Eigenvalue, eigenvector and Jordan normal form}
\\\noindent * We will consider a square matrix $A_{m\times m}$ s.t. $rank(A) = m$ in following discussion.\\[1ex]


  \noindent \textbf{[i] If $A$ have $m$ different Eigenvalue} 
\\\noindent Based on the discussion above, we know that it is the first step to analysis the $A^n$ to discribe all the linear system. Obviously, if A is a diagonal matrix, then the exponent of the matrix is easy and simple. 

\begin{theorem} Let $A$ is a diagonal matrix s.t. $A = diag(a_1, a_2, \ldots a_m)$, then $A^n = diag(a_1^n, a_2^n, \ldots a_m^n)$.
\end{theorem}

Furthermore, if matrix $A$ have $m$ different eigenvalue $\lambda_1, \lambda_2, \ldots, \lambda_m$ and $\mathbf v_i$ is the eigenvector of $\lambda_i$. Let
$$
\Lambda = diag(\lambda_1, \lambda_2, \ldots, \lambda_m), V = (\mathbf v_1, \mathbf v_2, \ldots \mathbf v_m)
$$
then we can easily prove that 
$$
A = V^{-1} \Lambda V
$$
And the calculation of $A^n$ is simple.
$$
A^n = V^{-1} \Lambda^n V = V^{-1} diag(\lambda_1^n, \lambda_2^n, \ldots, \lambda_m^n) V
$$

Now we back to consider the linear system, if $f(\mathbf x )= A \mathbf x$ and $A$ have $m$ different eigenvalue, then we know that 
$$
\mathbf x_{n} = A^n \mathbf x_0 = V^{-1} diag(\lambda_1^n, \lambda_2^n, \ldots, \lambda_m^n) V \mathbf x_0
$$

Based on the analysis in the section 1, we still want to analysis the convergence and divergence for ever system.
$$
\lim_{n \rightarrow \infty} x_{n} = V^{-1} diag(\lim_{n \rightarrow \infty}\lambda_1^n, \lim_{n \rightarrow \infty}\lambda_2^n, \ldots, \lim_{n \rightarrow \infty}\lambda_m^n) V \mathbf x_0
$$

Obviously, with the knowledge of sequence, if $|\lambda_i| \in [0, 1)$, then $\lim_{n \rightarrow \infty}\lambda_i^n = 0$ and the sequence is convergence. Also, if $|\lambda_i| \in (1, +\infty)$, then $\lim_{n \rightarrow \infty}\lambda_i^n = \infty$ and the sequence is divergence. So we have this conclusion.


\begin{theorem}\label{sink-source-saddle-linear-system}\textbf{Sink, source and saddle in linear system}
\\\noindent Consider a linear system $f(\mathbf x) = A \mathbf x$, where $A$ is a square matrix in $m$ dimension. If the eigenvalue of $A$ are $\lambda_1, \lambda_2, \ldots \lambda_m$ and
\\\noindent \textbf{[i]} $\forall i \in 1, 2, \ldots, m, |\lambda_i| < 1$, then the origin point is sink.
\\\noindent \textbf{[ii]} $\forall i \in 1, 2, \ldots, m, |\lambda_i| > 1$, then the origin point is source.
\\\noindent \textbf{[iii]} $\{i | |\lambda_i| < 1\} \neq \varnothing \land \{j | |\lambda_j| > 1\} \neq \varnothing$, that means, if at least one eigenvalue are absolute smaller than one and at least one is upper than one, then the origin point is saddle.\\
\end{theorem}





  \noindent \textbf{[ii] If $A$ have at least two equal eigenvalue} 
\\\noindent We can transfrom the matrix $A$ with Jordan normal form rather than eigenvalue diagonal matrix.
\\\noindent Consider the matrix $A_{m\times m}$ and the eigenvalue $\lambda_{1}, \lambda_{2}, \ldots, \lambda_{k}$ are $r_1, r_2, \ldots r_k$ multiple root of function $|\lambda I - A| = 0$, which satisfied the definition of eigenvalue, and $k < m$, $\sum_{i = 1}^{k} r_i = m$, $I = diag(1, 1, 1, \ldots, 1)$. 
\\\noindent Then for every $r_i$ multiple eigenvalue $\lambda_i$, $\exists \mathbf v_{i1}, \mathbf v_{i2}, \ldots \mathbf v_{i r_i}$ s.t.
$$
|\lambda I - A|\mathbf v_{i1} = 0, |\lambda I - A|\mathbf v_{ij+1} = \mathbf v_{ij}(j = 1, 2, \ldots r_i - 1)
$$
We can still structure the $V$ matrix same as $V$ in \textbf{[i]}, and we can also represent the diagonal eigenvalue matrix $\Lambda$ to the \textbf{ Jordan normal form matrix} $J$ which satisfied 
$$
J = \left[
\begin{array}{cccc}
J_{1} \\
& J_{2} \\
& & \ldots \\
& & & J_{k}
\end{array}
\right] = diag(J_1, J_2, \ldots J_k) \text{, where } J_i =\left[
\begin{array}{ccccc}
\lambda_i & 1 \\
& \lambda_i & 1 \\
& & \ldots & \ldots \\
& & & \lambda_i & 1 \\
& & & & \lambda_i
\end{array} \right]
$$
is $r_i$ dimension square matrix called \textbf{Jordan block}

Based on the calculation of block matrix we found that
$$
A^n = V^{-1} J^n V = V^{-1} diag(J_1^n, J_2^n, \ldots, J_k^n) V
$$

So familiar with the discussion in \textbf{[i]}, now it is necessary to discuss the $J_i^n$. On the other hand, we know that for ever Jordan block, we have
$$
J_i^n = \left[
\begin{array}{ccccc}
\lambda_i^n & (^n_1)\lambda_i^{n-1} & (^n_2)\lambda_i^{n-2} & \ldots      & (^n_{r_i})\lambda_i^{n-r_i}        \\
            & \lambda_i^n           & (^n_1)\lambda_i^{n-1} & \ldots      & (^n_{r_{i-1}})\lambda_i^{n-r_i+1}  \\
            &                       & \ldots                & \ldots      & \ldots                             \\
            &                       &                       & \lambda_i^n & (^n_1)\lambda_i^{n-1}              \\
            &                       &                       &             & \lambda_i^n
\end{array} \right]
$$ 

Obviously, for ever element on the diagonal, the Theo. \ref{sink-source-saddle-linear-system} still established. To proved that, we will prove the following theorem firstly.

\begin{theorem} \label{exp_Jordan}Let $J_i$ is a Jordan block with eigenvalue $\lambda_i$. 
\\\noindent \textbf{[i]} If $|\lambda_i| < 1$, then $\lim_{n \rightarrow \infty} J_i^n = 0$
\\\noindent \textbf{[ii]} If $|\lambda_i| > 1$, then $\lim_{n \rightarrow \infty} J_i^n = \infty$
\end{theorem}
{\color{blue}
\begin{proof} Consider a element $(^n_k)\lambda_i^{n-k}$ of $J_i$, then 
$$
\lim_{n \rightarrow \infty}(^n_k)\lambda_i^{n-k} = \lim_{n \rightarrow \infty}\left({n(n-1)\ldots(n-k) \over 1\cdot 2 \cdot \ldots \cdot k} \lambda_i^{n-k}\right)
$$

As the ${n(n-1)\ldots(n-k) \over 1\cdot 2 \cdot \ldots \cdot k}$ is a polynomial of $n$ in $k$ dimension, so $\exists a_1, a_2, \ldots a_k \in R$ s.t. 
$$
\lim_{n \rightarrow \infty}(^n_k)\lambda_i^{n-k} = \lim_{n \rightarrow \infty}\left(\sum_{p = 1}^{k} a_p n^p\right)\lambda_{n - k} = \sum_{p = 1}^{k} \lim_{n \rightarrow \infty}(a_p n^p \lambda_i^{n-k})
$$
Finally, we found, if $|\lambda_i| > 1$, then $\lim_{n \rightarrow \infty}(^n_k)\lambda_i^{n-k} = \infty$ and if $|\lambda_i| < 1$, then $\lim_{n \rightarrow \infty}(^n_k)\lambda_i^{n-k} = 0$ and the Theo. \ref{exp_Jordan} is established. $\blacksquare$
\end{proof}
}

\end{discussion}


As for the non-linear problem, a wildly used method is \textbf{Jacobian matrix}


\begin{definition}\textbf{Jacobian matrix}
\\\noindent Let $\mathbf f = (f_1, f_2, \ldots f_m)$ be a map on $R^m$ and $\mathbf p \in R^m$ is a point on $R^m$ space. The \textbf{Jacobian matrix} of $\mathbf f$ at $\mathbf p$ is the matrix 
$$
D\mathbf f(\mathbf p) = \left[
\begin{array}{cccc}
{\partial f_1 \over \partial x_1}(\mathbf p) & {\partial f_1 \over \partial x_2}(\mathbf p) & \ldots {\partial f_1 \over \partial x_m}(\mathbf p) \\
{\partial f_2 \over \partial x_1}(\mathbf p) & {\partial f_2 \over \partial x_2}(\mathbf p) & \ldots {\partial f_2 \over \partial x_m}(\mathbf p) \\
\ldots & \ldots & \ldots & \ldots \\
{\partial f_m \over \partial x_1}(\mathbf p) & {\partial f_m \over \partial x_2}(\mathbf p) & \ldots {\partial f_m \over \partial x_m}(\mathbf p) \\
\end{array} \right]
$$
\end{definition}

Jacobian matrix is a linearization estimation of a non-linear system that we can assume the derivative of the system near the point $\mathbf p$ is $D\mathbf f(\mathbf p)$. That means, instead of origin non-linear system, we can analysis the estimated system $\mathbf f_1(\mathbf x) = D\mathbf f(\mathbf p) \mathbf x$ where $x \in N(\mathbf p, \varepsilon)$ and $\varepsilon$ is a certain constant. Based on the Theo. \ref{sink-source-saddle-linear-system}, it is easy to improve the following conclusion.

\begin{theorem}\label{sink-source-saddle-Jacobian-matrix}\textbf{Sink, source and saddle in non-linear system}
\\\noindent Consider a non-linear system $\mathbf f(\mathbf x)$ and a fixed point $\mathbf p \in R^m$ s.t. $\mathbf f (\mathbf p) = \mathbf p$. If the Jacobian matrix of $\mathbf f$ at $\mathbf p$ is $D\mathbf f(\mathbf p)$, and $\Lambda$ are eigenvalue set of matrix $D\mathbf f(\mathbf p)$
\\\noindent \textbf{[i]} $\forall \lambda_i \in \Lambda  |\lambda_i| < 1$, then the $\mathbf p$ is a sink point.
\\\noindent \textbf{[ii]} $\forall \lambda_i \in \Lambda  |\lambda_i| > 1$, then the $\mathbf p$ is a source point.
\\\noindent \textbf{[iii]} $\{\lambda_i \in \Lambda | |\lambda_i| < 1\} \neq \varnothing \land \{\lambda_i \in \Lambda | |\lambda_i| > 1\} \neq \varnothing$, that means, if at least one eigenvalue are absolute smaller than one and at least one is upper than one, then the $\mathbf p$ is a saddle point.
\end{theorem}


Finally, we can analysis the property of fixed point in e.g. \ref{Henon-map-0-0*4}.
{\color{blue}
\begin{solution} We can consider the Henon map directly
$$
f(x, y) = (a - x^2 + by, x)  \Rightarrow Df(x, y) = \left[
\begin{array}{cc}
-2x & b \\
1 & 0 
\end{array} \right] 
$$
Let $\lambda$ are eigenvalue, then
$$
|\lambda I - Df(x, y)| = 0 \Rightarrow \left|
\begin{array}{cc}
-2x- \lambda & b \\
1 & -\lambda
\end{array} \right| = 0 \Rightarrow \lambda^2 + 2x\lambda - b = 0 \Rightarrow \lambda_{12} = {-x \pm\sqrt{x^2 + b}}
$$
When $(a, b) = (0, 0.4)$
$$
\text{If } (x, y) = (0, 0) \text{, then }|\lambda_{12}| = |-x\pm\sqrt{x^2 + b}| = |\sqrt{0.4}| < 1
$$
$$
\text{If } (x, y) = (0.6, 0.6) \text{, then }|\lambda_{12}| = |-x\pm\sqrt{x^2 + b}| = |-0.6 \pm \sqrt{0.76}| \Rightarrow |\lambda_1| = 1.472 > 1, |\lambda_2| = 0.272 < 1
$$
Finally, we proved that $(0, 0)$ is sink and $(-0.6, -0.6)$ is saddle just as what we found on simulation.$\blacksquare$


\end{solution}
}








\subsection{Stability and matrix periodic}
We found this conclusion based on the discussion above.

\begin{conclusion} The value of Jacobian matrix of a Henon map is just relevant to variable $x$ and parameter $b$. That means, if we reduce the dimension of parameter and fixed $b$ as $b_0$, then the property of fixed point will be determined only with variable $x$.
\end{conclusion}

As $y_{n+1} = x_n$, so we can just analysis the bifurcation of $a - x_{\infty}$ with random initial point.


If we consider the fixed point of system with arbitrary paremeter grou $(a, b)$, we found that the fixed point will satisfied 
\begin{flalign}
& x^2 + (1 - b)x - a = 0 \Rightarrow x = {1\over 2} (b-1) \pm\sqrt{(b-1)^2 + 4 a} & \label{formula-21}
\end{flalign}

and the fixed point is $(x, y) = ({1\over 2} (b-1) \pm\sqrt{(b-1)^2 + 4 a}, {1\over 2} (b-1) \pm\sqrt{(b-1)^2 + 4 a})$, so we have the Jacobian matrix at this fixed point as 
$$
Df(x, y) = \left[
\begin{array}{cc}
(b-1) \pm\sqrt{(b-1)^2 + 4 a} & b \\
1 & 0 
\end{array} \right] \text{, and the eigenvalue $\lambda_i$ satisfied }
$$
\begin{flalign}
& \lambda^2 - [(b-1) \pm\sqrt{(b-1)^2 + 4 a}]\lambda - b = 0 & \label{formula-22}
\end{flalign}

Then we can found the property of sink and source in every fixed point easily.

Now we focus on periodic-k orbit. Firstly, we still plot the bifurcation diagram of Henon map.

\begin{figure}[H]
\begin{center}
\includegraphics[width=0.6\textwidth]{figure/section2/Henon-orbit-bf-a--0*4.png}
\caption{Bifurcation diagram for Henon map ($b = 0.4$)}\label{Henon-map-b=0*4-BF}
\end{center}
\end{figure}



That is simple to analysis the influence of parameter. In following plots, $b \equiv 0.4$ and $a = 0.9, 0.988, 1.0, 1.0293, 1.045, 1.2$. We found $a = 0.9$ is a periodic-4 sink, $a = 0.988$ is a periodic-16 sink, $a = 1.0$ is a four-piece attractor, $a = 1.0293$ is a periodic-10 sink, $a = 1.045$ is two-piece attractor and the points of an orbit alternate between the pieces. Finally $a = 1.2$ two pieces have merged to form one-piece attractor.

\begin{definition}\textbf{Attractor} An attractor is a set of numerical values toward which a system tends to evolve, for a wide variety of starting conditions of the system.
\\\noindent Futuermore, in discrete time, we called the orbit of a system as periodic-k orbit. However in chaotic orbit, the solution set is a continuous (or uncountable) set. And we called this orbit as attractor.
\end{definition}

We will discuss the relationship between matrix and periodic-k orbit. But before that, it is necessary to introduce some new definition.

\begin{definition} A map $\mathbf f$ on $R^m$ is \textbf{one-to-one} if and only if $\mathbf f(\mathbf v_1) \mathbf f(\mathbf v_2) \Leftrightarrow \mathbf v_1 = \mathbf v_2$
\end{definition}





\begin{figure}[H]
\begin{minipage}[c][0.32\width]{0.32\textwidth}
   \centering
   \includegraphics[width=\textwidth]{figure/section2/Henon-attractor-0*9-0*4.png}
\end{minipage}
\begin{minipage}[c][0.32\width]{0.32\textwidth}
   \centering
   \includegraphics[width=\textwidth]{figure/section2/Henon-attractor-0*988-0*4.png}
\end{minipage}
\begin{minipage}[c][0.32\width]{0.32\textwidth}
   \centering
   \includegraphics[width=\textwidth]{figure/section2/Henon-attractor-1*0-0*4.png}
\end{minipage}
\\[12ex]
\begin{minipage}[c][0.32\width]{0.32\textwidth}
   \centering
   \includegraphics[width=\textwidth]{figure/section2/Henon-attractor-1*0293-0*4.png}
\end{minipage}
\begin{minipage}[c][0.32\width]{0.32\textwidth}
   \centering
   \includegraphics[width=\textwidth]{figure/section2/Henon-attractor-1*045-0*4.png}
\end{minipage}
\begin{minipage}[c][0.32\width]{0.32\textwidth}
   \centering
   \includegraphics[width=\textwidth]{figure/section2/Henon-attractor-1*2-0*4.png}
\end{minipage}
\\[4ex]\caption{Attractor of Henon map in different parameter}\label{attractor-Henon-map}
\end{figure}


\begin{definition}\textbf{Inverse map}
\\\noindent Consider a one-to-one map $\mathbf f$ on $R^m$. The inverse map $\mathbf f^{-1}$ is automatically exists and satisfied $\forall \mathbf v \in D \subset R^m, \mathbf f(\mathbf f^{-1})(\mathbf v) = \mathbf f^{-1}(\mathbf f)(\mathbf v) = \mathbf v$, where $D$ is domain of map.
\end{definition}

For instance, a one-to-one map $f(x) = 2x$ have an inverse map $f^{-1} = {x/2}$. Obviously, for every linear map $\mathbf f(\mathbf x) = A \mathbf x, \exists f^{-1}(\mathbf x) = A^{-1} \mathbf x$.

\begin{theorem} For every $R^m$ linear map $\mathbf f(\mathbf x) = A \mathbf x$ and $A$ s.t. $rank(A) = m$, the inverse map $f^{-1}$ always be existed.
\end{theorem}
{\color{blue}
\begin{proof}\textbf{[i]} If $A$ have $m$ different eigenvalue, then $A = V\Lambda V$ where $\Lambda = diag(\lambda_1, \lambda_2, \ldots, \lambda_m), V = (\mathbf v_1, \mathbf v_2, \ldots ,\mathbf v_m)$ where $\lambda_i$ is eigenvalue and $\mathbf v_1$ is eigenvector. Then 
$$
A^{-1} = \left(V^{-1} \Lambda V\right)^{-1} = V \Lambda^{-1} V^{-1} = V diag\left({1 \over \lambda_1}, {1 \over \lambda_2}, \ldots, {1 \over \lambda_m}\right) V^{-1}
$$
\\\noindent \textbf{[ii]} If $A$ have $p < m$ different eigenvalue, based on the Jordan normal form, we still have $J, V$ s.t. $A = V^{-1} J V$ and $J = diag(J_1, J_2, \ldots J_k)$ where $J_i$ is Jordan block based on the eigenvalue $\lambda_i$. And now the problem is prove that for evert Jordan block, the inverse block always be existed.
\\\noindent Obviously, $J_i = \lambda_i I + N$ where
$$
N = \left[
\begin{array}{ccccc}
0 & 1 & 0 & \ldots & 0 \\
0 & 0 & 1 & \ldots & 0 \\
\ldots & \ldots & \ldots & \ldots & \ldots \\
0 & 0 & 0 & \ldots & 1 \\
0 & 0 & 0 & \ldots & 0 \\
\end{array} \right]
$$

And it is simple to found that $N^m = \mathbf 0_{m\times m}$. Based on the Taylor expansion, we have 
$$
J_i^{-1} = \lambda_i^{-1} (I + \lambda_i^{-1} N + \lambda_i^{-2} N^2 - \ldots + (-\lambda_i)^{-n+1}N^{n-1})
$$

Although the inverse of Jordan block is not a Jordan block of $1/\lambda_i$, it is still exists and we proved the theorem. $\blacksquare$
\end{proof}
}







%~~~~~~~~~~~~~~~~~~~~~~~~~~~~~~~~~~~~~~~~~~~~~~~~~~~~~~~~~~~~~~~~~~~~~~~~~~~~~~~~~~~~~~~~~~~~~~~~~~~~~















%~~~~~~~~~~~~~~~~~~~~~~~~~~~~~~~~~~~~~~~~~~~~~~~~~~~~~~~~~~~~~~~~~~~~~~~~~~~~~~~~~~~~~~~~~~~~~~~~~~~~~
\newpage
\section{Chaos}

We discussed the Henon map in last section. However, different from the section 1, Logistic map has been wildly used in application problems, we talked less about why we are interested in this model. So, in this section, we will mainly introduce the motivation.




\subsection{Lorenz system, Henon map and Poincare section}



\begin{discussion} \textbf{Why are we intersted in Henon map?}
\\\noindent First of all, it is necessary to introduce a continuous model. The Lorenz system is a system of ordinary differential equations which notable for having chaotic solutions for certain parameter values and initial conditions. In particular, the Lorenz attractor is a set of chaotic solutions of the Lorenz system.

\begin{problem}\textbf{Lorenz model}
\\\noindent Lorenz model is a system of three ordinary differential equations now known as the Lorenz equations:
$$
{dx \over dt} = \sigma(y - x)
$$
$$
{dy \over dt} = x(\rho - z) - y
$$
$$
{dz \over dt} = xy - \beta z
$$
where $\sigma, \rho, \beta$ are parameters.
\end{problem}

It is continuous problem, however, based on the knowledge in numerical analysis, we can discrete the continuous to discrete problem in several ways.\footnote{EDWARD N LORENZ’S 1963 PAPER, “DETERMINISTIC NONPERIODIC FLOW”, IN JOURNAL OF THE ATMOSPHERIC SCIENCES, VOL 20, PAGES 130–141} 

We can reconstruct the Lorenz equation, or a normal continuous dynamical system as 
$$
{dX_i\over dt} = F_i(X_1, X_2, \ldots X_m), i = 1, 2, \ldots m
$$
which is a m-dim dynamical system and $t$ is single independence variable. To simplify this problem, we choose a initial time $t_0$ and time increment $\Delta t$, then let 
$$
X_{i, n} = X_i (t_0 + n \Delta t)
$$
we have several ways to aproximate the equations.



{\color{blue}
\begin{solution}\textbf{[i] Auxiliary approximations} $X_{i, n+1} = X_{i, n} + F_i(P_n)\Delta t$
\\\noindent \textbf{[ii] Centered difference procedure} $X_{i, n+1} = X_{i, n-1} + 2F_i(P_n)\Delta t$
\\\noindent \textbf{[iii] Double-approximation procedure} $X_{i, n+1} = X_{i, n} + {1\over 2}(F_i(P_n) + F_i(P_{n+1}))\Delta t$
\end{solution}
}


Even solve this group of function directly is difficult, it is not difficult to find the numerical solution. \\[2ex]

\begin{figure}[H]
\begin{minipage}[c][0.29\width]{0.29\textwidth}
   \centering
   \includegraphics[width=\textwidth]{figure/section2/Lorenz.png}
\end{minipage}
\begin{minipage}[c][0.4\width]{0.4\textwidth}
   \centering
   \includegraphics[width=\textwidth]{figure/section2/Lorenz-z-t.png}
\end{minipage}
\begin{minipage}[c][0.29\width]{0.29\textwidth}
   \centering
   \includegraphics[width=\textwidth]{figure/section2/Lorenz-map.png}
\end{minipage}
\\[3ex]\caption{Lorenz system, $z-t$ map and Lorenz map}\label{Lorenz-map}
\end{figure}


No we consider the problem in one dimension. In the second part of Fig. \ref{Lorenz-map}, we plot the z-t figure of Lorenz model. 

Ok, we found that it is still difficult to discribe the $z-t$ figure. However, after the discussion of the logistic map $g(x) = 4x(1-x)$ as well as chaotic orbit, we know in most situation, we just care about the boundary of interval of a map. On the other hand, we found that the $z-t$ figure of Lorenz system is familiar with sine function, it is shaking during the time iteration. So if we just consider the maxinum (or the mininum) of this $z-t$ map, we can analysis the problem easier.




\begin{definition}\textbf{Lorenz map}
\\\noindent The function $z_{n+1} = f(z_n)$ satisfied the last of Fig. \ref{Lorenz-map} is called the Lorenz map. The map can be discribed in following steps.
\\\noindent \textbf{[i]} Find the $z-t$ function in Lorenz model.
\\\noindent \textbf{[ii]} The map $\{z_n\}$ is a point set which is the maxima of $z-t$ function. $z_{n+1} = f(z_n)$ where $z_{n}$ is a maxima point of $z-t$ function and $z_{n+1}$ is next maxima point of the function with growing of $t$.
\end{definition}




* The graph of Lorenz map is not actually a curve. It does have some trickness because it is not a well-defined function. However trickness is so small and there is so much to be gained by treating the graph as a curve, that we will simplt make this approximation keeping in mind that the sunsequence analysis is plausible.


%** By the way, the Lorenz map is also a kind of Poincare map.\footnote{Reference: Nonlinear Dynamics and Chaos With Applications to Physics, Biology, Chemistry, and Engineering, Steven H. Strogatz, Section 9.4, ISBN-13: 978-0813349107}


%\begin{definition}\textbf{Poincare map}
%\\\noindent The Poincare map $P$ is a mapping from $S\subset R^m$ to itself, obtained by following tarjectories from one intersection with $S$ to the next. If $x_n \in S$ denotes the k-th intersection, thne the Poincare map is defined by 
%$$
%x_{n+1} = P(x_n)
%$$ 

%Futhermore, if $x_0$ is a fixed point i.e. $P(x_0) = x_0$, then a tarjectory staring at $x_0$ returens to $x_0$ after time $T$ and is therefore a closed orbit for the original system ${dX_i\over dt} = F_i(X_1, X_2, \ldots X_m), i = 1, 2, \ldots m$
%\end{definition}



%\footnote{Hénon, M., 2021. A two-dimensional mapping with a strange attractor.}. And this is Henon's map.

As Lorenz map have no formula to discribe, it is very difficult to research that. However, in Lorenz's paper, he gave a correspondence to analysis the map, called tent map, which we has been introduced in the section 1
$$
x_{n+1} = \left\{
\begin{array}{ll}
2x_n                & x_n < {1/2} \\
\text{Undefined}    & x_n = {1/2} \\
2 - 2x_n            & x_n > {1/2}
\end{array}
\right.
$$

At least this is a piecewise continuous funcion with one discontinuous point $x = 1/2$. So we found the property of this map is not good enough to analysis. We hope the function is continuous in all domain. And we found if we try to remove this discontinuous point, then the $f'$ will satisfy $f'^{+}(1/2) = f'^{-}(1/2) = 1$. We found it is similar to the Logistic model and it seems we can discuss the property of Logistic map rather than Lorenz map. And we will explain why we can discuss the Logistic map instead of Lorenz map.

On the other hand, we found the Henon map also familiar with Logistic map in Fig. \ref{attractor-Henon-map}. The only different between Henon map and Logistic map is Henon map is fat and Logistic is thin, or a line. However if we change our parameter, like $b \rightarrow 0$, then we found \\[4ex]


\begin{figure}[H]
\begin{minipage}[c][0.32\width]{0.32\textwidth}
   \centering
   \includegraphics[width=\textwidth]{figure/section2/Henon-attractor-1*2-0*4.png}
\end{minipage}
\begin{minipage}[c][0.32\width]{0.32\textwidth}
   \centering
   \includegraphics[width=\textwidth]{figure/section2/Henon-attractor-1*2-0*2.png}
\end{minipage}
\begin{minipage}[c][0.32\width]{0.32\textwidth}
   \centering
   \includegraphics[width=\textwidth]{figure/section2/Henon-attractor-1*2-0*05.png}
\end{minipage}
\\[3ex]\caption{Attractor of Henon map in different $b$(0.4, 0.2, 0.05)}\label{Henon-map-b}
\end{figure}



So it \textbf{seems} we can analysis the Henon map instead of Lorenz map. But we still need more proof and in this section, we will try to solve these problems.



\end{discussion}









\subsection{Lyapunov exponents and Conjugacy}

\begin{definition} \textbf{Asymptotically periodic}
\\\noindent Consider map $f \in C^1(R^1)$. An orbit $\{x_1, x_2, \ldots x_n, \ldots\}$ is called asymptotically periodic if it convergence to a periodic orbit as $n \rightarrow \infty$, that means, $\exists \{y_1, y_2, \ldots y_k, y_1, y_2, \ldots y_km ,\ldots\}$ is a periodic orbit s.t.
$$
\lim_{n \rightarrow \infty} |x_n - y_n| = 0
$$
Also, wecalled these map as \textbf{eventually periodic} beacause their orbit is eventually lands on a periodic orbit.
\end{definition} 

For instance, the Lorenz map is a eventually periodic map. Because we found in the begining of the iteration, the map shaking in a wild interation (Just as 1-15 iterates in Fig. \ref{Lorenz-map}, second image) and after this period, the map become stable and it is convergent to the map in the 3rd image of Fig. \ref{Lorenz-map}, which we called that Lorenz map.\\[4ex]


\begin{figure}[H]
\begin{minipage}[c][0.29\width]{0.29\textwidth}
   \centering
   \includegraphics[width=0.7\textwidth]{figure/section3/asymptotically-periodic.png}
\end{minipage}
\begin{minipage}[c][0.4\width]{0.4\textwidth}
   \centering
   \includegraphics[width=\textwidth]{figure/section3/Lorenz-map.png}
\end{minipage}
\\[3ex]\caption{Asymptotically periodic, intro and example in Lorenz map}\label{Asymptotically-periodic}
\end{figure}

\newpage



There are several other maps with this property, for example, in section 1, we introduced the Logistic map $G(x) = 4x(1-x)$, with the initial condition $x_0 = 1/2$, we found after 2 iterates, is coincides with the fixed point $x = 0$.

Now we try to find a method to judge a map is asymptotically periodic to another periodic map. In the section 1, we introduced the stability test for periodic orbits (Theo. \ref{stability-test-for-periodic-orbits}), we called the limitation of value in Theo. \ref{chain-rule-period-orbit} as \textbf{Lyapunov number}.

\begin{definition} \textbf{Lyapunov number and Lyapunov exponent}.
\\\noindent Consider map $f \in C^1(R^1)$. Define \textbf{Lyapunov number} $L(x_1)$ as 
$$
L(x_1) = \lim_{n \rightarrow \infty}(\prod_{i = 1}^{n} |f'(x_i)|)^{1/n}
$$
and based on the logarithm function, we can define the Lyapunov exponent as
$$
h_f(x_1) = h(x_1) = \lim_{n \rightarrow \infty}{1\over n}\left[\sum_{i = 1}^{n} \ln(f'(x_i))\right]
$$
Notice that $h$ exists if and only if $L$ exists and is nonzero, also $\ln L = h$. 
\end{definition} 

Based on this Lyapunov exponent, we have this theorem.
\begin{theorem}\label{Lyapunov-exponent-asymptotically-periodic} Consider map $f \in C^1(R^1)$. If orbits $\{x_1, x_2, \ldots\}$ of $f$ satisfies $f'(x_i) \neq 0 \forall i \in N$ and it is asymptotically periodic to the periodic orbit ${y_1, y_2, \ldots}$, then two orbit have indentical Lyapunov exponents, assuming both exist.
\end{theorem}


{\color{blue}
\begin{proof}\textbf{[i]} If we consider a sequence $\{s_i\}$ s.t. $\lim_{i \rightarrow \infty0} s_i = s$, then 
$$
\forall \varepsilon > 0, \exists N_1 \in \mathcal N \text{ s.t. } \forall n > N_1, |s_n - s| < \varepsilon
$$
Now we consider the average of $\{s_i\}$, we found for this $\varepsilon$,
$$
\lim_{N \rightarrow \infty}{1 \over N}\left|\sum_{i = 1}^{N}s_i - s\right| 
= \lim_{N \rightarrow \infty} {1\over N}\sum_{i = 1}^{N}\left|s_i - s\right| 
= \lim_{N \rightarrow \infty} {1\over N}\left[\sum_{i = 1}^{N_1 - 1}\left|s_i - s\right| + \sum_{i = N_1}^{N}\left|s_i - s\right|\right] 
$$
$$
= \lim_{N \rightarrow \infty} {1\over N}\sum_{i = 1}^{N_1 - 1}\left|s_i - s\right| + \lim_{N \rightarrow \infty}{1\over N}\sum_{i = N_1}^{N}\left|s_i - s\right| 
\leq 0 + \lim_{N \rightarrow \infty}{1\over N - N_1}\sum_{i = N_1}^{N}\left|s_i - s\right|
< {N 0 N_1\over N - N_1}\varepsilon = \varepsilon 
$$
So we have this conclusion
$$
\forall \varepsilon > 0, \exists N_1 \in \mathcal N, \text{ s.t. } \forall n > N_1, \lim_{N \rightarrow \infty}{1 \over N}\left|\sum_{i = 1}^{N}s_i - s\right| < \varepsilon \Rightarrow \lim_{N \rightarrow \infty}{1\over N}\sum_{i =1}^{N}s_i = s
$$
\newpage
  \noindent \textbf{[ii]} Let $y_1$ is the fixed point (that means, $x_i$ asymptotically periodic to a periodic-1 orbit), then $\lim_{n \rightarrow \infty} x_n = y_1$. As $f \in C^1(R^1)$, then $f'$ is exists and $f'$ Riemann integrable (and of course, Lebesgue integrable), so we can exchange the order of integral(or differential of $f$) and limitation, then we have
$$
\lim_{n \rightarrow \infty}f'(x_n) = f'(\lim_{n \rightarrow \infty} x_n) = f'(y_1)
$$
On the other hand, as $\ln|x|$ is a continuous, monotony function for $x \in R^+$, then 
$$
\lim_{n\rightarrow \infty} \ln |f'(x_n)| = \ln \left|\lim_{n\rightarrow \infty}f'(x_n)\right| = \ln |f'(y_1)| 
\Rightarrow h(x_1) = \lim_{n \rightarrow \infty}{1\over n}\sum_{i = 1}^{n} \ln |f'(x_i)| = \ln|f'(y_1)| = h(y_1)
$$
            \textbf{[iii]} Now we assume $k > 1, k \in \mathcal N$, obviously, $y_1$ is fixed point of $f^k$, and 
$$
h_{f^k}(x_1) = \ln |(f^k)'(y_1)| = h_{f^k}(y_1)
$$
            Now we will prove $h_{f^k}(x_1) = {1\over k} h_{f}(x_1)$. Based on the definition, we know
$$
h_{f^k}(x_1) = \lim_{n \rightarrow \infty}{1 \over n} \sum_{i = 1}^{n}\ln\left|(f^k)'(x_i)\right| 
= \lim_{n \rightarrow \infty}{1 \over n} \sum_{i = 1}^{n}\ln\left|{1\over k}\prod_{j = i}^{i + k}f'(x_i)\right| 
= {1\over k}\lim_{n \rightarrow \infty}{1 \over n} \sum_{i = 1}^{n}\sum_{j = i}^{i + k}\ln\left|f'(x_i)\right| 
= {1 \over k} h_f(x_1)
$$
            And we proved the theorem. $\blacksquare$
\end{proof}
}


\begin{figure}[H]
\begin{center}
\includegraphics[width=0.5\textwidth]{figure/section3/logistic-lyapunov-exp.png} \\
\caption{Lyapunov exponent of logistic model in different parameter}\label{logistic-lyapunov-exp}
\end{center}
\end{figure}


Obviously, the chaotic orbit satisfied Def. \ref{chaos-orbit} will have no asymptotically periodic and we have this theorem.


\begin{theorem} Consider map $f \in C^1(R^1)$, the orbit is \textbf{chaotic} if the Lyapunov exponent $h(x_1)$ is greater than zero.
\end{theorem}

We can check this conclusion with Fig. \ref{logistic-lyapunov-exp}, however, we still need more strickly proof.




{\color{blue}
\begin{proof}$\blacksquare$
%[TODO] How to prove this theorem
\end{proof}
}



\newpage

Now we will discuss the tent map.
\begin{discussion}\textbf{Tent map and chaos}
\begin{theorem} The tent map $T$ has infinitely many chaotic orbits.
\end{theorem}
\begin{figure}[H]
\begin{center}
\includegraphics[width=1\textwidth]{figure/section3/tent-map.png} \\
\caption{Tent map}\label{tent-map}
\end{center}
\end{figure}



\end{discussion}









\end{document}
%~~~~~~~~~~~~~~~~~~~~~~~~~~~~~~~~~~~~~~~~~~~~~~~~~~~~~~~~~~~~~~~~~~~~~~~~~~~~~~~~~~~~~~~~~~~~~~~~~~~~~


















































\newpage
\newpage
\appendix
\section{Prove of periodic-3 orbit theory (Theo \ref{p3-chaos})}
Now we try to solve the problem we discussed in periodic-3 orbit of logistic model.


Before we solve this problem, we need some other definition.
\begin{definition}\textbf{Critical point}
\\\noindent Consider a function $f$ in domain $D$, we called every point $x \in D$ s.t. $f'(x) = 0$ or the derivative not exist as \textbf{critical point}.
\end{definition}
\begin{definition}\textbf{Unimodal}
\\\noindent Let $f$ is a map on $R$, if $f$ has and only has one critical point, then we called $f$ is \textbf{unimodal}.
\end{definition}

\begin{example} \label{e-g-p3-orbit}Consider a map in following figure, which is a periodic-3 orbit, denoted $\{A, B, C\}$ s.t. $f(A) = B, f(B) = C, f(C) = A$.\\[4ex]

\begin{figure}[H]
\begin{minipage}[c][0.24\width]{0.24\textwidth}
   \centering
   \includegraphics[width=\textwidth]{figure/section1/period-3-orbit.png}
\end{minipage}
\begin{minipage}[c][0.24\width]{0.24\textwidth}
   \centering
   \includegraphics[width=\textwidth]{figure/section1/period-3-orbit-map.png}
\end{minipage}
\begin{minipage}[c][0.24\width]{0.24\textwidth}
   \centering
   \includegraphics[width=\textwidth]{figure/section1/period-3-orbit-trans.png}
\end{minipage}
\begin{minipage}[c][0.24\width]{0.24\textwidth}
   \centering
   \includegraphics[width=\textwidth]{figure/section1/period-3-orbit-itinerary.png}
\end{minipage}
\\[6ex]\caption{An example periodic-3 orbit map and analysis}\label{period-3-orbit}
\end{figure}

There are some basic conclusion here
\begin{conclusion} \textbf{[i]} $A < B < C, f(A) = B, f(B) = C, f(C) = A$;
\\\noindent \textbf{[ii]} $f$ is unimodal;
\\\noindent \textbf{[iii]} $\forall x \in R \backslash [A, C], f(x) \notin [A, C]$, that means we can take no care of point out of the interval $[A, C]$.
\\\noindent \textbf{[iv]} Let $f([A, B]) = \{y | \exists x \in [A, B] \text{ s.t. } f(x) = y\}$, then $[B, C] \subset f([A, B])$ ({\color{red}\textbf{red}} line and subinterval in second image of Fig. \ref{period-3-orbit})
\\\noindent \textbf{[v]} $[A, B] = f([B, C])$ ({\color{Fuchsia} \textbf{purple}} line and subinterval in second image of Fig. \ref{period-3-orbit})
\end{conclusion}

\end{example}

\newpage 
Now we try to prove the Theo. \ref{p3-chaos} in e.g.\ref{e-g-p3-orbit}.
{\color{blue}
\begin{proof}\textbf{[i]} 

$\blacksquare$
\end{proof}
}









]