%Pretreatment========================================================================================
\documentclass[12pt]{article}
\usepackage{lingmacros}
\usepackage{tree-dvips}
\usepackage{graphicx}
\usepackage{hyperref}
\usepackage{amsmath}
\usepackage{amssymb}
\usepackage{multicol}
\usepackage{geometry}
\usepackage{cite}
\usepackage[amsmath,thmmarks]{ntheorem}
\usepackage{algpseudocode}
\usepackage{algorithm}
\usepackage{listings} 
\usepackage{verbatim}
\usepackage{subfigure}
\usepackage{appendix}  
\usepackage{color}
%\usepackage{subcaption}


\theoremstyle{plain}
\theoremseparator{\hspace{1em}} \theoremnumbering{arabic}
\theoremsymbol{}
\newtheorem{theorem}{\textbf{Theorem}}[section]
\newtheorem{definition}{\textbf{Definition}}[section]
\newtheorem{lemma}{\textbf{Lemma}}[section]
\newtheorem{proof}{\textit{PROOF}}[section]
\newtheorem{example}{\textbf{E x a m p l e}}[section]
\newtheorem{solution}{\textit{SOLUTION}}[section]
\newtheorem{discussion}{\textit{D I S C U S S I O N}}[section]

\geometry{left=2cm,right=2cm,top=3cm,bottom=2cm}
\title{[Note] Chaos: An Introduction to Dynamical Systems}
\author{}
\date{\today}

\begin{document}
\maketitle

{
\begin{center}
\LARGE \textbf{Problem in discrete-time system}
\end{center}
}
\section{One-Dimension Maps}

\begin{definition}\textbf{n-order differentiable function, Smooth function, Map}
\\\noindent Consider an open set $E$ and $n \in \mathcal N$, called 
$$
C^n(E) = \{f \in C(E) | \forall \alpha \text{ s.t. } |\alpha| \leq n, D^\alpha f \in C(E)\}
$$
is n-order differentiable function set of $E$, where $C(E)$ is continuous function on $E$.
\\\noindent If $f$ on domian $E$ have infinity-order derivative, or $f \in C^\infty(E)$, then called $f$ \textbf{smooth function}.
\\\noindent If the function $f$ have same domain and range, then called $f$ is a \textbf{map}.
\end{definition}

{\color{red} The fucntion in this book will be a smooth function if we not emphasize.}

\begin{definition}\textbf{Orbit, initial value, fixed point}
\\\noindent Consider a map $f: X \rightarrow X$, $x$ is a point in $X$ then 
\\\noindent Called \textbf{orbit} of $x$ is a set of point 
$$
\text{Orbit}(X) = \{x, f(x), f^2(x), \ldots f^n(x), \ldots\}
$$
, where $f^n(x) = f(f(\ldots f(x))) = (f\circ f \circ f \circ \ldots \circ f)(x)$.
\\\noindent The starting point of $x$ for a orbit called the \textbf{initial value}.
\\\noindent If the point $p$ s.t. $f(p) = p$, then called $p$ as fixed point.
\end{definition}

OK, and now we consider two dynamical systems, with a input $x$, the system will always return to $f(x) = 2x$ and $g(x) = 2x(1-x)$. And then the output will become the input value and etc. During this looping, it is simple to find the orbit of a certain initial value.


\begin{table}[H]
\centering  
\caption{Comparison of exponential growth and logistic growth}  
\begin{tabular}{|c||c|c|c|c|c|c|c|c|c|c|}
\hline
f & init & 1 & 2 & 3 & 4 & 5 & 6 & 7 & 8 & 9 \\
\hline
\hline
$f(x)$ & 0.01 & 0.02   & 0.04   & 0.08   & 0.16   & 0.32     & 0.64        & 1.28  & 2.56  & 5.12  \\
\hline
$f(x)$ & 0.01 & 0.0198 & 0.0388 & 0.0746 & 0.138  & 0.268    & 0.362       & 0.462 & 0.497 & 0.499 \\
\hline
$g(x)$ & 0.5  & 0.5    & 0.5    & 0.5    & 0.5    & 0.5      & 0.5         & 0.5   & 0.5   & 0.5   \\
\hline
$g(x)$ & 0.8  & 0.32   & 0.435  & 0.492  & 0.499  & 0.5      & 0.5         & 0.5   & 0.5   & 0.5   \\
\hline
$g(x)$ & 1.2  & -0.48  & -1.42  & -6.87  & -108.4 & -23716.9 & -1125030476 & -inf  & -inf  & -inf  \\
\hline
\end{tabular}  
\end{table}  

We found that in the model of $f$, the result is growth as exponential function and we called that exponential growth. Also, when intial value $x \in [0, 1]$, with iteration, the result have limition of 0.5 and we called these model as logistic growth.

In this section, we will mainly focus on these kind of dynamical system, obviously, the iteration processing of model are discrete, we also called these dynamical system models as \textbf{maps}.






\subsection{Cobweb plot, stability and }
To analysis a maps, the basic method is based on cobweb plot. Fig \ref{cobweb-plow-1} showed a method to analysis a dynamical system with a certain iteration principle. 
\begin{figure}[H]
\begin{center}
\includegraphics[width=0.6\textwidth]{figure/section1/cobweb-plot.png} \\
\caption{An example of cobweb plot and basic principle}\label{cobweb-plow-1}
\end{center}
\end{figure}

In every iteration, the independent and dependent variable exchanged their location and we can found a group of $\{x_1, y_2, x_3, y_4, \ldots\}$ as orbit of initial value. Or, with the symmetric line $y = x$, it is simple to symmetric all black line to purple line and we can build a cobweb plot with origin image and $y = x$ to find a group of $\{x_1, x_2, \ldots\}$ as orbit from the initial value.

Before we discuss the different of fixed point, it is necessary to review some basic definitions.

\begin{definition}\textbf{$\varepsilon$ Neighbourhood}
\\\noindent In a metric space $X$, an $\varepsilon$ neighbourhood $N_\varepsilon(p)$ of point $p$ is defined 
$$
N_\varepsilon(p) = \{x \in X | d(x, p) < \varepsilon\}
$$
where $d(x, p)$ is the distance bewteen point $p$ and $x$. Also, in a $R^1$ space, the $\varepsilon$ Neighbourhood is give by
$$
N_\varepsilon(p) = \{x \in R | |x-p| < \varepsilon\}
$$
\end{definition}


\newpage

\begin{figure}[H]
\begin{minipage}[c][0.6\width]{
   0.5\textwidth}
   \centering
   \includegraphics[width=1\textwidth]{figure/section1/cb01.png}
\end{minipage}
\begin{minipage}[c][0.6\width]{
   0.5\textwidth}
   \centering
   \includegraphics[width=1\textwidth]{figure/section1/cb02.png}
\end{minipage}
\begin{minipage}[c][0.6\width]{
   0.5\textwidth}
   \centering
   \includegraphics[width=1\textwidth]{figure/section1/cb03.png}
\end{minipage}
\begin{minipage}[c][0.6\width]{
   0.5\textwidth}
   \centering
   \includegraphics[width=1\textwidth]{figure/section1/cb04.png} \\
\end{minipage}
\caption{Cobweb plot in different initial value}\label{logistic-cobweb-plot}
\end{figure}


It is simple to find that for all initial value $x \in (0, 1)$, with iteration, the output have limitation in 0.5. On the other hand, to solve the equation $x = 2x(1-x)$ we found $x_1 = 0, x_2 = 1, x_3 = 0.5$ as three fixed point. So we have two kinds of fixed point, the one is limitation point and the other is not.


\begin{definition}\textbf{Sink, Source (Attracting and Repelling Fixed Point)}
\\\noindent Consider a map $f: R \rightarrow R$ and point $p$ s.t. $f(p) = p$, then
\\\noindent If for evert points sufficiently to $p$ are attracted to $p$, then called $p$ as \textbf{sink}, or \textbf{attracting fixed point}. Or
$$
\text{For an } \varepsilon > 0, \forall x \in N_\varepsilon(p), \lim_{k \rightarrow \infty}f^k(x) = p \text{ then called } p \text{ as \textbf{sink}}.
$$
If for every points sufficiently to $p$ are repelled to $p$, then called $p$ as \textbf{source}, or \textbf{repelling fixed point}.
$$
\text{For an } \varepsilon > 0, \forall x \in N_\varepsilon(p), \lim_{k \rightarrow \infty}f^k(x) = p \text{ then called } p \text{ as \textbf{sink}}.
$$
\end{definition}



\newpage

\begin{theorem} \label{sink-source-point}Let $f$ is a map on $R$, assume $p$ is a fixed point of $f$, then
\\\noindent [i] If $|f'(p)| < 1$, then $p$ is a sink;
\\\noindent [ii] If $|f'(p)| > 1$, then $p$ is a source.
\end{theorem}


{\color{blue}
\begin{proof} \textbf{[i]} Based on definition of derivative, we have
$$
\lim_{x \rightarrow p} {|f(x) - f(p)| \over |x - p|} = |f'(p)|
$$
Now. let $a \in (\min(|f'(p)|, 1), \max(|f'(p)|, 1))$ (e.g. $a = {1\over 2}(1+ |f'(p)|)$), then
$$
\forall a \in (\min(|f'(p)|, 1), \max(|f'(p)|, 1)), \exists \varepsilon_0 > 0 \text{ s.t. } \forall \varepsilon \in (0, \varepsilon_0], \forall x \in N_\varepsilon(p), {|f(x) - f(p)| \over |x - p|} < a
$$
            \textit{That means, $f(x)$ is closer to $p$ than $x$ (or distant bewteen curve $y = f(x)$ and $y = x$), but at least a factor of $a$ and we have the conclusion
$$
\forall x \in N_\varepsilon(p), f(x) \in N_\varepsilon(p)
$$
            During the iteration processing it is simple to find that all orbit $\{f(x), f^2(x), \ldots f^n(x), \ldots\} \subset N_\varepsilon(p)$, so now we can consider another conclusion in follow.}
\\[2ex]\noindent \textbf{[ii]} We try to prove the inequality $\forall x \in N_\varepsilon(p), |f^k(x) - p|$
\\\noindent \textbf{[ii-1]} Obvious, if $k = 1$, then $|f(x) - p| = |f(x) - f(p)| < a |x-p|$ ($p$ is fixed point so $f(p) = p$)
\\\noindent \textbf{[ii-2]} If $k = 2$, Based on the conclusion in \textbf{[i]}, $x_1 = f(x) \in N_\varepsilon(p)$ and $|f(x_1) - p| < a |x_1 - p| < a^2 |x - p|$
\\\noindent $\ldots$
\\\noindent \textbf{[ii-k+1]} (Assume the inequality is established in $k$), then
$$
|f^{k+1}(x) - p| < a|f^k(x) - p| < a \cdot a^k |x - p| = a^{k+1}|x - p|
$$
            In summary, for all $k \in N$, the inequality is established.
\\[2ex]\noindent \textbf{[iii-1]} Now we consider the equality condition, if $|f'(p)| < 1$ then $a < 1$ and  
$$
\lim_{k \rightarrow \infty}|f^{k}(x) - p| < |x-p|\lim_{k \rightarrow \infty} a^k = 0 \text{ (Because }a \in (0, 1)\text{)} 
$$
            So we have the conclusion, 
$$
\forall x \in N_\varepsilon(p), \lim_{k \rightarrow \infty}f^k(x) = p
$$
            \textbf{[iii-2]} Also, if $|f'(p)| > 1$ then $a^k \rightarrow \infty$, that means, with the iteration, the maps will eventually outside the condition, or the domain interval. $\blacksquare$
\end{proof}
}

* We will discuss what happened while $f'(p) = 1$ laterly.

** Obviously, this theorem expressed a kind of convergence, as the speed of the convergence is based on the $a$ in exponent function, we called this convergence as \textbf{Exponential Convergence}.

Now we consider another map as example.


\newpage
\begin{example} Solved the fixed point of $\varphi(x) = (3x -x^2)/2$, find every sink and source point with Theo. \ref{sink-source-point}
\end{example}


\begin{figure}[H]
\begin{center}
\includegraphics[width=0.5\textwidth]{figure/section1/cobweb-plot-2.png} \\
%\caption{}\label{cobweb-plow-2}
\end{center}
\end{figure}

{\color{blue}
\begin{solution}
It is simple to find the fixed point with $x = (3x - x^3) / 2$ and $x_1 = 1, x_2 = 0, x_3 = -1$. Based on the image, we can found that $1$ and $-1$ are sink and $0$ is source. On the other hand
$$
\varphi'(x) = {3\over 2} (1 - x^2), \varphi'(-1) = 0 < 1, \varphi'(0) = {3 \over 2} > 1, \varphi'(1) = 0 < 1
$$ 
and we proved the conclusion we found on figure before. $\blacksquare$
\end{solution}
}

Another way to confirm a point is sink or source is based on the formula identity and algebra. For instance, we consider the distance bewteen $g(x) = 2x(1-x)$ and fixed point $1/2$, then 
$$
|g(x) - 1/2| = |2x(1-x) - 1/2| = 2|x - 1/2||x - 1/2|
$$

and $\forall x \in (0, 1), |x - 1/2| < 1 \Rightarrow |g(x) - 1/2| < 1$, that means the distance bewteen $g(x)$ and $p$ is decreasing during time iteration and we can confirm that $1/2$ is a sink point rather than source point.

Next, we will focus on a logistic model with different parameter.








\subsection{Periodic points, family of logistic maps}
\begin{example} Find the fixed point of $g(x) = 3.3x(1-x), x \in [0, 1]$.
\end{example}

{\color{blue}
\begin{solution}
It is simple to find the fixed point with $x = 3.3x(1-x)$ and $x_1 = 0, x_2 = 23/33, x_3 = 1$. Obviously, both 0 and 1 are source. And
$$
g'(x) = 3.3 - 6.6x, |g'(23/33)| = 1.3 > 1
$$ 
So all these three fixed point are source, and it is simple to find the conclusion with cobweb plot. $\blacksquare$
\end{solution}
}

\newpage
\begin{figure}[H]
\begin{center}
\includegraphics[width=0.6\textwidth]{figure/section1/periodic-point.png} \\
\caption{An example of periodic point}\label{periodic-point}
\end{center}
\end{figure}


Hold on a second, something strange! Even we cannot find a sink fixed point, all of initial value are sank into a group of points!

\begin{definition}\textbf{Period-k point, Period-k orbit}
\\\noindent Let $f$ be a map on $R$, and $p$ is a point in domain, if $f^k(p) = p$, and $k$ is the smallest such positive integer, then called $p$ as \textbf{periodic point of period $k$}, or \textbf{period-k point};
\\\noindent Called orbit with initial point $p$ as \textbf{periodic orbit of period k}, or \textbf{period-k orbit};
\end{definition}

\begin{definition}\textbf{Sink and Source in Period point}
\\\noindent Let $f$ be a map and $p$ is a period-k point
\\\noindent If $p$ is a sink, then called this period-k orbit as periodic sink;
\\\noindent If $p$ is a source, then called this period-k orbit as periodic source.
\end{definition}

Obviously, based on the chain rule, we have $(fg)'(x) = f'(g(x))g'(x)$, let $f = g, x = p_1$, then 
$$
g^2(p_1) = g'(g(p_1))g'(p_1) = g'(p_2)g'(p_1)
$$
Summary this formula, we have 
\begin{theorem} \label{chain-rule-period-orbit}For every map $f$ and period-k orbit $\{p_1, p_2, \ldots p_k\}$, 
$$
(f^k)'(p_1) = (f^k)'(p_2) = \ldots = (f^k)'(p_k) = \prod_{i = 1}^{k}f'(p_i)
$$
\end{theorem}

{\color{blue}
\begin{proof} \textbf{Theo. \ref{chain-rule-period-orbit}} 
$$
(f^k)(p_1) = (f(f^{k-1}))'(p_1) = f'(f^{k-1}(p_1))(f^{k-1})'(p_1) = \ldots = \prod_{i = 1}^{k}f'(p_i) = (f^k)(p_i) (\forall i = 1, 2, \ldots, k)
$$
And we proved the theorem. $\blacksquare$   
\end{proof}
}

\newpage
Same as Theo. \ref{sink-source-point}, we have stability test for periodic orbits.
\begin{theorem} \textbf{Stability test for periodic orbits}
\\\noindent Let $f$ is a map and period-k orbit $\{p_1, p_2, \ldots p_k\}$, 
\\\noindent If $|\prod_{i = 1}^{k}f'(p_i)| < 1$ then called this periodic orbit is a sink;
\\\noindent If $|\prod_{i = 1}^{k}f'(p_i)| > 1$ then called this periodic orbit is a source;
\end{theorem}


{\color{blue}
\begin{proof} \textbf{Theo. \ref{sink-source-point}} 
\\\noindent Consider a new map $g(x) = f^k(x)$, where $f$ be a map and $p$ is a period-k point, then $p$ is a fixed point of $g$. Based on Theo. \ref{sink-source-point}, $|g(p)| < 1$ if $p$ is sink and $|g(p)| > 1$ if $p$ is a source. On the other hand, $g(p) = f^k(p) = \prod_{i = 1}^{k}f'(p_i)$ and we proved the theorem. $\blacksquare$   
\end{proof}
}

Now we consider another problem.

\begin{example} Find the fixed point or periodic orbit of $g_{3.5}(x) = 3.5x(1-x), g_{3.86}(x) = 3.86x(1-x), x \in [0, 1]$.
\end{example}


\begin{figure}[H]
\begin{minipage}[c][0.5\width]{
   0.5\textwidth}
   \centering
   \includegraphics[width=0.9\textwidth]{figure/section1/logistic35.png}
\end{minipage}
\begin{minipage}[c][0.5\width]{
   0.5\textwidth}
   \centering
   \includegraphics[width=0.9\textwidth]{figure/section1/logistic386.png} \\
\end{minipage}
\caption{Logistic maps in $a = 3.5$ and $a = 3.86$}\label{logistic-no-periodic}
\end{figure}

We found in $a = 3.5$, even the periodic orbit is difficult to find, the iteration still have a boundary. If we consider every $a \in [1, 4]$, we can plot a figure between parameter $a$ and orbits $x$, and the figure was made by following repearting:
\\\noindent \textbf{[i]} Choose a value $a$, starting with $a = 1$.
\\\noindent \textbf{[ii]} Choose a value $x \in [0, 1]$ randomly.
\\\noindent \textbf{[iii]} Calculate the orbit of x under $g_a(x)$ in a certain iteration times $t_max$.
\\\noindent \textbf{[iv]} Ignore the first $t_0$ iterates and plot the orbit.


\begin{figure}[H]
\begin{center}
\includegraphics[width=0.8\textwidth]{figure/section1/logistic-stability.png} \\
\caption{Logistic model stability interval ($a \in [1, 4]$)}\label{Logistic-stability}
\end{center}
\end{figure}




\newpage
\begin{discussion} Now we will discuss the family of logistic maps with Fig. \ref{Logistic-stability}
\\\noindent \textbf{[i] Periodic-3 window}
\\\noindent We found periodic-1 orbits (or point) and periodic-2 orbits, based on the image above, it seem we also have periodic-3 orbits. And now we focus on the interval of parameter $a$ rather than domain of function, we found there is a interval of $a$ inside the $[3.83, 3.86]$ and we called these kind of interval as ``periodic window''. For instance, next figure showed the periodic-3 window of $a$.

\begin{figure}[H]
\begin{minipage}[c][0.5\width]{
   0.5\textwidth}
   \centering
   \includegraphics[width=0.70\textwidth]{figure/section1/periodic-3-window.png}
\end{minipage}
\begin{minipage}[c][0.5\width]{
   0.5\textwidth}
   \centering
   \includegraphics[width=0.9\textwidth]{figure/section1/logistic384.png} \\
\end{minipage}
\caption{Periodic-3 window and cobweb plot in $a = 3.84$}\label{logistic-cobweb-plot1}
\end{figure}

That's fine, let's check the result by cobweb plot. Ok, hold on a second, something wrong! So we still need more analysis.

Obviously, every periodic-3 orbit of $g$ is a fixed point of $g^3$, so we can also analysis $g^3$ map.\\[3ex]

\begin{figure}[H]
\begin{minipage}[c][0.33\width]{0.33\textwidth}
   \centering
   \includegraphics[width=\textwidth]{figure/section1/g3logistic-origin.png}
\end{minipage}
\begin{minipage}[c][0.33\width]{0.33\textwidth}
   \centering
   \includegraphics[width=\textwidth]{figure/section1/g3logistic384.png}
\end{minipage}
\begin{minipage}[c][0.33\width]{0.33\textwidth}
   \centering
   \includegraphics[width=\textwidth]{figure/section1/g3logistic384-001-detail.png} 
\end{minipage}
\\[3ex]\caption{$g^3$ map and $g^3$ cobweb plot figure}\label{logistic-cobweb-plot1}
\end{figure}


\newpage
  \noindent \textbf{[ii] The Logistic Map $G(x) = 4x(1-x)$}
\\\noindent Now we consider another logistic map where $a \equiv 4$. \\[1ex]

\begin{figure}[H]
\begin{minipage}[c][0.24\width]{0.24\textwidth}
   \centering
   \includegraphics[width=\textwidth]{figure/section1/4-logistic-ite-1.png}
\end{minipage}
\begin{minipage}[c][0.24\width]{0.24\textwidth}
   \centering
   \includegraphics[width=\textwidth]{figure/section1/4-logistic-ite-2.png}
\end{minipage}
\begin{minipage}[c][0.24\width]{0.24\textwidth}
   \centering
   \includegraphics[width=\textwidth]{figure/section1/4-logistic-ite-3.png}
\end{minipage}
\begin{minipage}[c][0.24\width]{0.24\textwidth}
   \centering
   \includegraphics[width=\textwidth]{figure/section1/4-logistic-ite-4.png}
\end{minipage}
\\[3ex]\caption{$g^1_4, g_2^4, g_3^4$ and $g_4^4$ figure}\label{4-logistic-ite}
\end{figure}


\end{discussion}
\end{document}
%~~~~~~~~~~~~~~~~~~~~~~~~~~~~~~~~~~~~~~~~~~~~~~~~~~~~~~~~~~~~~~~~~~~~~~~~~~~~~~~~~~~~~~~~~~~~~~~~~~~~~






















