%Pretreatment========================================================================================
\documentclass[12pt]{article}
\usepackage{lingmacros}
\usepackage{tree-dvips}
\usepackage{graphicx}
\usepackage{hyperref}
\usepackage{amsmath}
\usepackage{amssymb}
\usepackage{multicol}
\usepackage{geometry}
\usepackage{cite}
\usepackage[amsmath,thmmarks]{ntheorem}
\usepackage{algpseudocode}
\usepackage{algorithm}
\usepackage{listings} 
\usepackage{verbatim}
\usepackage{subfigure}
\usepackage{appendix}  
\usepackage{color}
\usepackage{wrapfig}
\usepackage[dvipsnames]{xcolor}
%\usepackage{subcaption}


\theoremstyle{plain}
\theoremseparator{\hspace{1em}} \theoremnumbering{arabic}
\theoremsymbol{}
\newtheorem{theorem}{\textbf{Theorem}}[section]
\newtheorem{definition}{\textbf{Definition}}[section]
\newtheorem{lemma}{\textbf{Lemma}}[section]
\newtheorem{proof}{\textit{PROOF}}[section]
\newtheorem{example}{\textbf{E x a m p l e}}[section]
\newtheorem{solution}{\textit{SOLUTION}}[section]
\newtheorem{discussion}{\textit{D I S C U S S I O N}}[section]
\newtheorem{conclusion}{\textit{\textbf{CONCLUSION}}}[section]

 
\geometry{left=2cm,right=2cm,top=3cm,bottom=2cm}
\title{[Note] Chaos: An Introduction to Dynamical Systems}
\author{}
\date{\today}

\begin{document}
\maketitle

{
\begin{center}
\LARGE \textbf{Problem in discrete-time system}
\end{center}
}
\section{One-Dimension Maps}

\begin{definition}\textbf{n-order differentiable function, Smooth function, Map}
\\\noindent Consider an open set $E$ and $n \in \mathcal N$, called 
$$
C^n(E) = \{f \in C(E) | \forall \alpha \text{ s.t. } |\alpha| \leq n, D^\alpha f \in C(E)\}
$$
is n-order differentiable function set of $E$, where $C(E)$ is continuous function on $E$.
\\\noindent If $f$ on domian $E$ have infinity-order derivative, or $f \in C^\infty(E)$, then called $f$ \textbf{smooth function}.
\\\noindent If the function $f$ have same domain and range, then called $f$ is a \textbf{map}.
\end{definition}

{\color{red} The fucntion in this book will be a smooth function if we not emphasize.}

\begin{definition}\textbf{Orbit, initial value, fixed point}
\\\noindent Consider a map $f: X \rightarrow X$, $x$ is a point in $X$ then 
\\\noindent Called \textbf{orbit} of $x$ is a set of point 
$$
\text{Orbit}(X) = \{x, f(x), f^2(x), \ldots f^n(x), \ldots\}
$$
, where $f^n(x) = f(f(\ldots f(x))) = (f\circ f \circ f \circ \ldots \circ f)(x)$.
\\\noindent The starting point of $x$ for a orbit called the \textbf{initial value}.
\\\noindent If the point $p$ s.t. $f(p) = p$, then called $p$ as fixed point.
\end{definition}

OK, and now we consider two dynamical systems, with a input $x$, the system will always return to $f(x) = 2x$ and $g(x) = 2x(1-x)$. And then the output will become the input value and etc. During this looping, it is simple to find the orbit of a certain initial value.


\begin{table}[H]
\centering  
\caption{Comparison of exponential growth and logistic growth}  
\begin{tabular}{|c||c|c|c|c|c|c|c|c|c|c|}
\hline
f & init & 1 & 2 & 3 & 4 & 5 & 6 & 7 & 8 & 9 \\
\hline
\hline
$f(x)$ & 0.01 & 0.02   & 0.04   & 0.08   & 0.16   & 0.32     & 0.64        & 1.28  & 2.56  & 5.12  \\
\hline
$f(x)$ & 0.01 & 0.0198 & 0.0388 & 0.0746 & 0.138  & 0.268    & 0.362       & 0.462 & 0.497 & 0.499 \\
\hline
$g(x)$ & 0.5  & 0.5    & 0.5    & 0.5    & 0.5    & 0.5      & 0.5         & 0.5   & 0.5   & 0.5   \\
\hline
$g(x)$ & 0.8  & 0.32   & 0.435  & 0.492  & 0.499  & 0.5      & 0.5         & 0.5   & 0.5   & 0.5   \\
\hline
$g(x)$ & 1.2  & -0.48  & -1.42  & -6.87  & -108.4 & -23716.9 & -1125030476 & -inf  & -inf  & -inf  \\
\hline
\end{tabular}  
\end{table}  

We found that in the model of $f$, the result is growth as exponential function and we called that exponential growth. Also, when intial value $x \in [0, 1]$, with iteration, the result have limition of 0.5 and we called these model as logistic growth.

In this section, we will mainly focus on these kind of dynamical system, obviously, the iteration processing of model are discrete, we also called these dynamical system models as \textbf{maps}.






\subsection{Cobweb plot, stability and }
To analysis a maps, the basic method is based on cobweb plot. Fig \ref{cobweb-plow-1} showed a method to analysis a dynamical system with a certain iteration principle. 
\begin{figure}[H]
\begin{center}
\includegraphics[width=0.6\textwidth]{figure/section1/cobweb-plot.png} \\
\caption{An example of cobweb plot and basic principle}\label{cobweb-plow-1}
\end{center}
\end{figure}

In every iteration, the independent and dependent variable exchanged their location and we can found a group of $\{x_1, y_2, x_3, y_4, \ldots\}$ as orbit of initial value. Or, with the symmetric line $y = x$, it is simple to symmetric all black line to purple line and we can build a cobweb plot with origin image and $y = x$ to find a group of $\{x_1, x_2, \ldots\}$ as orbit from the initial value.

Before we discuss the different of fixed point, it is necessary to review some basic definitions.

\begin{definition}\textbf{$\varepsilon$ Neighbourhood}
\\\noindent In a metric space $X$, an $\varepsilon$ neighbourhood $N_\varepsilon(p)$ of point $p$ is defined 
$$
N_\varepsilon(p) = \{x \in X | d(x, p) < \varepsilon\}
$$
where $d(x, p)$ is the distance bewteen point $p$ and $x$. Also, in a $R^1$ space, the $\varepsilon$ Neighbourhood is give by
$$
N_\varepsilon(p) = \{x \in R | |x-p| < \varepsilon\}
$$
\end{definition}


\newpage

\begin{figure}[H]
\begin{minipage}[c][0.6\width]{
   0.5\textwidth}
   \centering
   \includegraphics[width=1\textwidth]{figure/section1/cb01.png}
\end{minipage}
\begin{minipage}[c][0.6\width]{
   0.5\textwidth}
   \centering
   \includegraphics[width=1\textwidth]{figure/section1/cb02.png}
\end{minipage}
\begin{minipage}[c][0.6\width]{
   0.5\textwidth}
   \centering
   \includegraphics[width=1\textwidth]{figure/section1/cb03.png}
\end{minipage}
\begin{minipage}[c][0.6\width]{
   0.5\textwidth}
   \centering
   \includegraphics[width=1\textwidth]{figure/section1/cb04.png} \\
\end{minipage}
\caption{Cobweb plot in different initial value}\label{logistic-cobweb-plot}
\end{figure}


It is simple to find that for all initial value $x \in (0, 1)$, with iteration, the output have limitation in 0.5. On the other hand, to solve the equation $x = 2x(1-x)$ we found $x_1 = 0, x_2 = 1, x_3 = 0.5$ as three fixed point. So we have two kinds of fixed point, the one is limitation point and the other is not.


\begin{definition}\textbf{Sink, Source (Attracting and Repelling Fixed Point)}
\\\noindent Consider a map $f: R \rightarrow R$ and point $p$ s.t. $f(p) = p$, then
\\\noindent If for evert points sufficiently to $p$ are attracted to $p$, then called $p$ as \textbf{sink}, or \textbf{attracting fixed point}. Or
$$
\text{For an } \varepsilon > 0, \forall x \in N_\varepsilon(p), \lim_{k \rightarrow \infty}f^k(x) = p \text{ then called } p \text{ as \textbf{sink}}.
$$
If for every points sufficiently to $p$ are repelled to $p$, then called $p$ as \textbf{source}, or \textbf{repelling fixed point}.
$$
\text{For an } \varepsilon > 0, \forall x \in N_\varepsilon(p), x \neq p, \lim_{k \rightarrow \infty}f^k(x) \notin N_\varepsilon(x) \text{ then called } p \text{ as \textbf{source}}.
$$
\end{definition}



\newpage

\begin{theorem} \label{sink-source-point}Let $f$ is a map on $R$, assume $p$ is a fixed point of $f$, then
\\\noindent [i] If $|f'(p)| < 1$, then $p$ is a sink;
\\\noindent [ii] If $|f'(p)| > 1$, then $p$ is a source.
\end{theorem}


{\color{blue}
\begin{proof} \textbf{[i]} Based on definition of derivative, we have
$$
\lim_{x \rightarrow p} {|f(x) - f(p)| \over |x - p|} = |f'(p)|
$$
Now. let $a \in (\min(|f'(p)|, 1), \max(|f'(p)|, 1))$ (e.g. $a = {1\over 2}(1+ |f'(p)|)$), then
$$
\forall a \in (\min(|f'(p)|, 1), \max(|f'(p)|, 1)), \exists \varepsilon_0 > 0 \text{ s.t. } \forall \varepsilon \in (0, \varepsilon_0], \forall x \in N_\varepsilon(p), {|f(x) - f(p)| \over |x - p|} < a
$$
            \textit{That means, $f(x)$ is closer to $p$ than $x$ (or distant bewteen curve $y = f(x)$ and $y = x$), but at least a factor of $a$ and we have the conclusion
$$
\forall x \in N_\varepsilon(p), f(x) \in N_\varepsilon(p)
$$
            During the iteration processing it is simple to find that all orbit $\{f(x), f^2(x), \ldots f^n(x), \ldots\} \subset N_\varepsilon(p)$, so now we can consider another conclusion in follow.}
\\[2ex]\noindent \textbf{[ii]} We try to prove the inequality $\forall x \in N_\varepsilon(p), |f^k(x) - p|$
\\\noindent \textbf{[ii-1]} Obvious, if $k = 1$, then $|f(x) - p| = |f(x) - f(p)| < a |x-p|$ ($p$ is fixed point so $f(p) = p$)
\\\noindent \textbf{[ii-2]} If $k = 2$, Based on the conclusion in \textbf{[i]}, $x_1 = f(x) \in N_\varepsilon(p)$ and $|f(x_1) - p| < a |x_1 - p| < a^2 |x - p|$
\\\noindent $\ldots$
\\\noindent \textbf{[ii-k+1]} (Assume the inequality is established in $k$), then
$$
|f^{k+1}(x) - p| < a|f^k(x) - p| < a \cdot a^k |x - p| = a^{k+1}|x - p|
$$
            In summary, for all $k \in N$, the inequality is established.
\\[2ex]\noindent \textbf{[iii-1]} Now we consider the equality condition, if $|f'(p)| < 1$ then $a < 1$ and  
$$
\lim_{k \rightarrow \infty}|f^{k}(x) - p| < |x-p|\lim_{k \rightarrow \infty} a^k = 0 \text{ (Because }a \in (0, 1)\text{)} 
$$
            So we have the conclusion, 
$$
\forall x \in N_\varepsilon(p), \lim_{k \rightarrow \infty}f^k(x) = p
$$
            \textbf{[iii-2]} Also, if $|f'(p)| > 1$ then $a^k \rightarrow \infty$, that means, with the iteration, the maps will eventually outside the condition, or the domain interval. $\blacksquare$
\end{proof}
}

* We will discuss what happened while $f'(p) = 1$ laterly.

** Obviously, this theorem expressed a kind of convergence, as the speed of the convergence is based on the $a$ in exponent function, we called this convergence as \textbf{Exponential Convergence}.

Now we consider another map as example.


\newpage
\begin{example} Solved the fixed point of $\varphi(x) = (3x -x^2)/2$, find every sink and source point with Theo. \ref{sink-source-point}
\end{example}


\begin{figure}[H]
\begin{center}
\includegraphics[width=0.5\textwidth]{figure/section1/cobweb-plot-2.png} \\
%\caption{}\label{cobweb-plow-2}
\end{center}
\end{figure}

{\color{blue}
\begin{solution}
It is simple to find the fixed point with $x = (3x - x^3) / 2$ and $x_1 = 1, x_2 = 0, x_3 = -1$. Based on the image, we can found that $1$ and $-1$ are sink and $0$ is source. On the other hand
$$
\varphi'(x) = {3\over 2} (1 - x^2), \varphi'(-1) = 0 < 1, \varphi'(0) = {3 \over 2} > 1, \varphi'(1) = 0 < 1
$$ 
and we proved the conclusion we found on figure before. $\blacksquare$
\end{solution}
}

Another way to confirm a point is sink or source is based on the formula identity and algebra. For instance, we consider the distance bewteen $g(x) = 2x(1-x)$ and fixed point $1/2$, then 
$$
|g(x) - 1/2| = |2x(1-x) - 1/2| = 2|x - 1/2||x - 1/2|
$$

and $\forall x \in (0, 1), |x - 1/2| < 1 \Rightarrow |g(x) - 1/2| < 1$, that means the distance bewteen $g(x)$ and $p$ is decreasing during time iteration and we can confirm that $1/2$ is a sink point rather than source point.

Next, we will focus on a logistic model with different parameter.








\subsection{Periodic points, family of logistic maps}
\begin{example} Find the fixed point of $g(x) = 3.3x(1-x), x \in [0, 1]$.
\end{example}

{\color{blue}
\begin{solution}
It is simple to find the fixed point with $x = 3.3x(1-x)$ and $x_1 = 0, x_2 = 23/33, x_3 = 1$. Obviously, both 0 and 1 are source. And
$$
g'(x) = 3.3 - 6.6x, |g'(23/33)| = 1.3 > 1
$$ 
So all these three fixed point are source, and it is simple to find the conclusion with cobweb plot. $\blacksquare$
\end{solution}
}

\newpage
\begin{figure}[H]
\begin{center}
\includegraphics[width=0.6\textwidth]{figure/section1/periodic-point.png} \\
\caption{An example of periodic point}\label{periodic-point}
\end{center}
\end{figure}


Hold on a second, something strange! Even we cannot find a sink fixed point, all of initial value are sank into a group of points!

\begin{definition}\textbf{Period-k point, Period-k orbit}
\\\noindent Let $f$ be a map on $R$, and $p$ is a point in domain, if $f^k(p) = p$, and $k$ is the smallest such positive integer, then called $p$ as \textbf{periodic point of period $k$}, or \textbf{period-k point};
\\\noindent Called orbit with initial point $p$ as \textbf{periodic orbit of period k}, or \textbf{period-k orbit};
\end{definition}

\begin{definition}\textbf{Sink and Source in Period point}
\\\noindent Let $f$ be a map and $p$ is a period-k point
\\\noindent If $p$ is a sink, then called this period-k orbit as periodic sink;
\\\noindent If $p$ is a source, then called this period-k orbit as periodic source.
\end{definition}

Obviously, based on the chain rule, we have $(fg)'(x) = f'(g(x))g'(x)$, let $f = g, x = p_1$, then 
$$
g^2(p_1) = g'(g(p_1))g'(p_1) = g'(p_2)g'(p_1)
$$
Summary this formula, we have 
\begin{theorem} \label{chain-rule-period-orbit}For every map $f$ and period-k orbit $\{p_1, p_2, \ldots p_k\}$, 
$$
(f^k)'(p_1) = (f^k)'(p_2) = \ldots = (f^k)'(p_k) = \prod_{i = 1}^{k}f'(p_i)
$$
\end{theorem}

{\color{blue}
\begin{proof} \textbf{Theo. \ref{chain-rule-period-orbit}} 
$$
(f^k)(p_1) = (f(f^{k-1}))'(p_1) = f'(f^{k-1}(p_1))(f^{k-1})'(p_1) = \ldots = \prod_{i = 1}^{k}f'(p_i) = (f^k)(p_i) (\forall i = 1, 2, \ldots, k)\blacksquare
$$
\end{proof}
}

\newpage
Same as Theo. \ref{sink-source-point}, we have stability test for periodic orbits.
\begin{theorem} \textbf{Stability test for periodic orbits}
\\\noindent Let $f$ is a map and period-k orbit $\{p_1, p_2, \ldots p_k\}$, 
\\\noindent If $|\prod_{i = 1}^{k}f'(p_i)| < 1$ then called this periodic orbit is a sink;
\\\noindent If $|\prod_{i = 1}^{k}f'(p_i)| > 1$ then called this periodic orbit is a source;
\end{theorem}


{\color{blue}
\begin{proof} \textbf{Theo. \ref{sink-source-point}} 
\\\noindent Consider a new map $g(x) = f^k(x)$, where $f$ be a map and $p$ is a period-k point, then $p$ is a fixed point of $g$. Based on Theo. \ref{sink-source-point}, $|g(p)| < 1$ if $p$ is sink and $|g(p)| > 1$ if $p$ is a source. On the other hand, $g(p) = f^k(p) = \prod_{i = 1}^{k}f'(p_i) \blacksquare$   
\end{proof}
}

Now we consider another problem.

\begin{example} Find the fixed point or periodic orbit of $g_{3.5}(x) = 3.5x(1-x), g_{3.86}(x) = 3.86x(1-x), x \in [0, 1]$.
\end{example}


\begin{figure}[H]
\begin{minipage}[c][0.5\width]{
   0.5\textwidth}
   \centering
   \includegraphics[width=0.9\textwidth]{figure/section1/logistic35.png}
\end{minipage}
\begin{minipage}[c][0.5\width]{
   0.5\textwidth}
   \centering
   \includegraphics[width=0.9\textwidth]{figure/section1/logistic386.png} \\
\end{minipage}
\caption{Logistic maps in $a = 3.5$ and $a = 3.86$}\label{logistic-no-periodic}
\end{figure}

We found in $a = 3.5$, even the periodic orbit is difficult to find, the iteration still have a boundary. If we consider every $a \in [1, 4]$, we can plot a figure between parameter $a$ and orbits $x$, and this \textbf{bifurcation diagram} was made by following repearting:
\\\noindent \textbf{[i]} Choose a value $a$, starting with $a = 1$.
\\\noindent \textbf{[ii]} Choose a value $x \in [0, 1]$ randomly.
\\\noindent \textbf{[iii]} Calculate the orbit of x under $g_a(x)$ in a certain iteration times $t_max$.
\\\noindent \textbf{[iv]} Ignore the first $t_0$ iterates and plot the orbit.


\begin{figure}[H]
\begin{center}
\includegraphics[width=0.8\textwidth]{figure/section1/logistic-stability.png} \\
\caption{Logistic model stability interval ($a \in [1, 4]$)}\label{Logistic-stability}
\end{center}
\end{figure}




\newpage
\begin{discussion} Now we will discuss the family of logistic maps with Fig. \ref{Logisstic-stability}
\\\noindent \textbf{[i] Periodic-3 window}
\\\noindent We found periodic-1 orbits (or point) and periodic-2 orbits, based on the image above, it seem we also have periodic-3 orbits. And now we focus on the interval of parameter $a$ rather than domain of function, we found there is a interval of $a$ inside the $[3.83, 3.86]$ and we called these kind of interval as ``periodic window''. For instance, next figure showed the periodic-3 window of $a$.

\begin{figure}[H]
\begin{minipage}[c][0.5\width]{
   0.5\textwidth}
   \centering
   \includegraphics[width=0.70\textwidth]{figure/section1/periodic-3-window.png}
\end{minipage}
\begin{minipage}[c][0.5\width]{
   0.5\textwidth}
   \centering
   \includegraphics[width=0.9\textwidth]{figure/section1/logistic384.png} \\
\end{minipage}
\caption{Periodic-3 window and cobweb plot in $a = 3.84$}\label{logistic-cobweb-plot1}
\end{figure}

That's fine, let's check the result by cobweb plot. Ok, hold on a second, something wrong! So we still need more analysis.

Obviously, every periodic-3 orbit of $g$ is a fixed point of $g^3$, so we can also analysis $g^3$ map.\\[3ex]

\begin{figure}[H]
\begin{minipage}[c][0.33\width]{0.33\textwidth}
   \centering
   \includegraphics[width=\textwidth]{figure/section1/g3logistic-origin.png}
\end{minipage}
\begin{minipage}[c][0.33\width]{0.33\textwidth}
   \centering
   \includegraphics[width=\textwidth]{figure/section1/g3logistic384.png}
\end{minipage}
\begin{minipage}[c][0.33\width]{0.33\textwidth}
   \centering
   \includegraphics[width=\textwidth]{figure/section1/g3logistic384-001-detail.png} 
\end{minipage}
\\[3ex]\caption{$g^3$ map and $g^3$ cobweb plot figure}\label{logistic-cobweb-plot1}
\end{figure}

We found different from periodic-2 orbit, the periodic-3 orbit is nearby(rather than equal) the point and it seems we have periodic-3 orbit. Actually, we will explain all periodic-3 will implies a characteristic we called ``chaos''. 






\newpage
  \noindent \textbf{[ii] The Logistic Map $G(x) = 4x(1-x)$}
\\\noindent Now we consider another logistic map where $a \equiv 4$. \\[1ex]
\\\noindent Firstly, why we are interested in $g_4(x)$, consider a quadratic function 
$$
g_a(x) = ax(1-x) = a(-x^2 + x - 1 + 1) = -a(x-{1\over 2})^2 +{a\over 4}
$$
this function have maximum at point $x = 1/2$ and the maximum is $a/4$. As we have the Theo. \ref{sink-source-point}, if we consider the sink point set, it is necessary to satisfy $|g_a(x)| < 1$, or${a \over } 4 < 1 \Rightarrow a < 4$. So at the point $a = 4$, this set is empty and this is a critical state. For every $a_{new} = a - \varepsilon (\varepsilon \rightarrow 0)$, we have the interval of sink. So at this point, some special property has been result and that is why we interested in this map.

We can still find the fixed point of $g_4(x)$ to solve $g_4(x) = x$, and we have $x_{11} = 0, x_{21} = {3/4}$. If we consider periodic-k orbit, for instance, we consider periodic-2 orbit, then we have solve the function $g(g(x)) = x$ as 
$$
g(g(x)) = 4(4x(1-x))[1-4x(1-x)] = x \Rightarrow (4x^2-4x+1)(x-1)x+{x\over 16} = 0 
$$
$$
\Rightarrow (4x-3)(16x^2 - 20x+5)x = 0 \Rightarrow x_{21} = 0, x_{22} = {3\over 4}, x_{23,24} = {5\pm \sqrt{5}\over 8}
$$

Also, it is easy to check the periodic-k orbit in the figure.\\[2ex]

\begin{figure}[H]
\begin{minipage}[c][0.24\width]{0.24\textwidth}
   \centering
   \includegraphics[width=\textwidth]{figure/section1/4-logistic-ite-1.png}
\end{minipage}
\begin{minipage}[c][0.24\width]{0.24\textwidth}
   \centering
   \includegraphics[width=\textwidth]{figure/section1/4-logistic-ite-2.png}
\end{minipage}
\begin{minipage}[c][0.24\width]{0.24\textwidth}
   \centering
   \includegraphics[width=\textwidth]{figure/section1/4-logistic-ite-3.png}
\end{minipage}
\begin{minipage}[c][0.24\width]{0.24\textwidth}
   \centering
   \includegraphics[width=\textwidth]{figure/section1/4-logistic-ite-4.png}
\end{minipage}
\\[3ex]\caption{$g_4^1, g_4^2, g_4^3$ and $g_4^4$ figure}\label{4-logistic-ite}
\end{figure}

\begin{figure}[H]
\begin{center}
\includegraphics[width=0.5\textwidth]{figure/section1/4-logistic-cobweb-plot.png}\\
\caption{$g_4(x)$ cobweb plot(periodic-1,2 orbits)}\label{4-logistic-ite}
\end{center}
\end{figure}

\newpage
We found a conclusion here
\begin{conclusion}
For every periodic-k, the model $g_4^k$ have $2^{k} - 1$ saddle-node bifurcation and $2^k$ fixed point. And these $2^k$ points include every fixed point for model $g_4^{i}, i = 1, 2, \ldots, k-1 \land k \equiv 0 (\text{mod } i)$
\end{conclusion}

The number of orbits of the map for each period can be tabulated in the map's periodic table.



\begin{table}[H]
\centering  
\caption{The periodic table for the logistic4 map}  
\begin{tabular}{|c||c|c|c|c|c|c|c|}
\hline
Period $k$                             & 1 & 2 & 3 & 4  & 5  & 6  & 7   \\
\hline
\hline
Number of fixed points of $g_4^k$      & 2 & 4 & 8 & 16 & 32 & 64 & 128 \\
\hline
Orbits of Period $k$                   & 2 & 1 & 2 & 3  & 4  & 5  & 6   \\
\hline
Fixed points due to lower orbits       & 0 & 2 & 2 & 4  & 2  & 4  & 2   \\
\hline
\hline
1                                      & / &   &   &    &    &    &   \\
2                                      & * & / &   &    &    &    &   \\
3                                      & * &   & / &    &    &    &   \\
4                                      & - & * &   & /  &    &    &   \\
5                                      & * &   &   &    & /  &    &   \\
6                                      & - & * & * &    &    & /  &   \\
7                                      & * &   &   &    &    &    & / \\
\hline
\end{tabular}  
\end{table}
(*: Greatest common divisor group, -: $g^k$ fixed)
\end{discussion}







\subsection{Chaos}
We still focus on $g_4(x)$ map, we try to check the $g_4^2$ fixed point ${5 - \sqrt{5}\over 8}$.\\[2ex]

\begin{figure}[H]
\begin{minipage}[c][0.24\width]{0.24\textwidth}
   \centering
   \includegraphics[width=\textwidth]{figure/section1/4-logistic-stable-30.png}
\end{minipage}
\begin{minipage}[c][0.24\width]{0.24\textwidth}
   \centering
   \includegraphics[width=\textwidth]{figure/section1/4-logistic-stable-50.png}
\end{minipage}
\begin{minipage}[c][0.24\width]{0.24\textwidth}
   \centering
   \includegraphics[width=\textwidth]{figure/section1/4-logistic-stable-100.png}
\end{minipage}
\begin{minipage}[c][0.24\width]{0.24\textwidth}
   \centering
   \includegraphics[width=\textwidth]{figure/section1/4-logistic-stable-500.png}
\end{minipage}
\\[3ex]\caption{$g_4^1, g_4^2, g_4^3$ and $g_4^4$ figure}\label{4-logistic-ite}
\end{figure}

It seems something wrong. Because we proved that ${5 \pm \sqrt{5}\over 8}$ is a periodic orbit during the iteration, but once we growth the iteration times, the results filled all the interval.

So what happened? We try to put all of our data into a same image, and we have
\begin{figure}[H]
\begin{center}
\includegraphics[width=0.8\textwidth]{figure/section1/4-logistic-stable-check.png} \\
\caption{Iteration and ``periodic-2 orbit'' value}\label{periodic-2-check}
\end{center}
\end{figure}

Obviously, in about first 40 times iteration, it was worked for a while, but with the iteration increasing, the error also increaed rapidly. Ok, ok, let's check the data for more details.
\begin{table}[H]
\centering  
\caption{Logistic4 periodic-2 orbit iteration}  
\begin{tabular}{|c||c|c|c|c|}
\hline
1-4   & 0.3454915028125262  & 0.9045084971874737  & 0.3454915028125262  & 0.9045084971874735 \\
\hline
5-8   & 0.34549150281252694 & 0.9045084971874745  & 0.3454915028125237  & 0.9045084971874705 \\
\hline
9-12  & 0.34549150281253665 & 0.9045084971874865  & 0.3454915028124849  & 0.9045084971874225 \\
\hline
13-16 & 0.34549150281269186 & 0.9045084971876783  & 0.3454915028118641  & 0.9045084971866552 \\
\hline
17-20 & 0.3454915028151752  & 0.9045084971907479  & 0.34549150280193086 & 0.9045084971743771 \\
\hline
21-24 & 0.3454915028549078  & 0.9045084972398602  & 0.3454915026430002  & 0.904508496977928  \\
\hline
25-27 & 0.3454915034906304  & 0.9045084980256565  & 0.34549150010010987 & $\ldots$           \\
\hline
\end{tabular}  
\end{table}

We noticed that during the iteration, the values of periodic-2 orbit are actually changed very small. Then we realized that is beacause of ${5 - \sqrt{5}\over 8} \neq 0.3454915028125262$ and this is just a value near the periodic point.(And the computer can only calculate this estimation value rather than real value.) Even this two value are almost nearby, it still have a little difference, and this difference become larger and larger during the iteration.

That is important beacause we found even two value are almost equal, after iterate, this tiny, tiny difference will become a catastrophe and eventually two orbits move apart.

\begin{definition}\textbf{Sensitive dependence on initial conditions, Sensitive point}
\\\noindent Let $f$ is a map on $R$, $x_0$ in domain.
\\\noindent If there is a nonzero distance $d$ s.t. some points arbitrary near $x_0$ are eventually mapped at least $d$ units from the corresponding image of $x_0$, then we called $x_0$ has \textbf{sensitive dependence on initial conditions};
\\\noindent If for this $x_0, \exists \varepsilon > 0$ s.t. $\forall x \in N_\varepsilon^o(x_0) = N_\varepsilon(x_0) \backslash \{x_0\}, \exists K$ s.t. $\forall k > K, ||f^k(x) - f^k(x_0)|| \geq \varepsilon$, then called this point is \textbf{sensitive point}.
\end{definition}
\begin{definition}\textbf{Eventually periodic}
\\\noindent Let $f$ is a map on $R$, $x_0$ in domain. If for some positive integer $N, \forall n > N, f^{n+p}(x) = f^n(x)$, then we called $x$ \textbf{eventually periodic} with period p, where p is the smallest such positive integer.
\end{definition}

Now we consider another model to explain this definition in another way.

\begin{example}Consider a map $f(x) = 3x (\text{mod } 1)$. (e.g. $f(4.33) = 0.33, f(-1.98) = 0.02$.)
\end{example}

\begin{figure}[H]
\begin{center}
\includegraphics[width=0.4\textwidth]{figure/section1/3xmod1.png} \\
\caption{3x mod 1 cobweb plot(initial value: {\color{green}0.25(green)}, {\color{red}0.2501(red)})}\label{3xmod1}
\end{center}
\end{figure}

\newpage
Basically, we have
\begin{theorem} For any map $f$, the source has sensitive dependence on initial conditions.
\end{theorem}
{\color{blue}
\begin{proof} For a certain $\varepsilon$, as $p$ is a source, then $\forall x \in N^o_\varepsilon(p), \lim_{k \rightarrow \infty}f^k(x) \notin N^\varepsilon_d(p) \Rightarrow d(p, x) > \varepsilon$ $\blacksquare$   
\end{proof}
}

Is there any way to investigate this sensitive dependence? Yes, and here we will introduce a method called \textbf{itinerary} of an orbit.

{\color{blue}
\begin{solution}\textbf{Itinerary}
\\\noindent We still consider the $g_4$ model. Assign the symbol \textbf{L} to the left subinterval $[0, 1/2]$ and \textbf{R} to the right subinterval $[1/2, 1]$. Then, for every initival condition $x_0$, we can list the itinerary with \textbf{L} and \textbf{R}. 
\\\noindent For instance, the initial point $x_0 = 1/3$ have the itinerary \textbf{LRLRLRRLLRR}
\begin{table}[H]
\centering  
\caption{Logistic4 $1/3$ itinerary}  
\begin{tabular}{|c||c|c|c|}
\hline
0-2   & 0.333333333333333  (\textbf{L}) & 0.888888888888889  (\textbf{R})  & 0.39506172839506154 (\textbf{L}) \\
\hline
3-5   & 0.9559518366102727 (\textbf{R}) & 0.16843169076687667(\textbf{L})  & 0.5602498252491516 (\textbf{R})  \\
\hline
6-8   & 0.9854798342297868 (\textbf{R}) & 0.05723732222487492 (\textbf{L}) & 0.21584484467760304(\textbf{L})  \\
\hline
9-10  & 0.6770233908148179 (\textbf{R}) & 0.874650876417697   (\textbf{R}) & $\ldots$                         \\
\hline
\end{tabular}  
\end{table}

And we can list all itinerary with different initial value.$\blacksquare$
\begin{table}[H]
\centering  
\caption{Logistic4 itinerary with different initial value}  
\begin{tabular}{|c||l|l|l|l|l|}
\hline
Val  & $1-10$              & $11-20$             & $21-30$             & $31-40$             & $\ldots$ \\
\hline
\hline
0.01 & \textbf{LLLRRLLLLR} & \textbf{RRRLRLRLRR} & \textbf{LLRRRLLRRR} & \textbf{RLRRLRLRLR} & $\ldots$ \\
\hline
0.25 & \textbf{LRRRRRRRRR} & \textbf{RRRRRRRRRR} & \textbf{RRRRRRRRRR} & \textbf{RRRRRRRRRR} & $\ldots$ \\
\hline
1/3  & \textbf{LRLRLRRLLR} & \textbf{RLRLLRRRRL} & \textbf{LLLRRRRLRL} & \textbf{RRRRRRRLRR} & $\ldots$ \\
\hline
0.5  & \textbf{RRLLLLLLLL} & \textbf{LLLLLLLLLL} & \textbf{LLLLLLLLLL} & \textbf{LLLLLLLLLL} & $\ldots$ \\
\hline
1    & \textbf{RLLLLLLLLL} & \textbf{LLLLLLLLLL} & \textbf{LLLLLLLLLL} & \textbf{LLLLLLLLLL} & $\ldots$ \\
\hline
\end{tabular}  
\end{table}

\end{solution}
}




Notice that there are some conclusions.
\begin{conclusion} For every periodic-k point, the itinerary of orbit will repeats \textbf{L} or \textbf{R} infinitely.
\end{conclusion}

\begin{conclusion} For every $k$ iterate, the itinerary have $2^k$ choice and the sum of their lengths is 1(or the length of the interval).
\end{conclusion}

Also, we have a conclusion not very obvious.

\begin{conclusion} Each $2^k$ itinerary is shorter than $\pi / 2^{k+1}$.
\end{conclusion}
We will prove this conclusion in later sections.



\newpage
We can also analysis the problem with \textbf{transition graph}.

\begin{figure}[H]
\begin{center}
\includegraphics[width=0.2\textwidth]{figure/section1/transition-graph-1.png} \\
\caption{Transition graph}\label{transition-graph-1}
\end{center}
\end{figure}

Finally, we focus on the title of this subsection ``chaos'', after these analysis, it is simple to summary the definition of chaos.

\begin{definition}\textbf{Chaos}
\\\noindent A chaotic orbit is a bounded, non-periodic orbit that displays sensitive dependence. Chaotic orbits seoarate exponentially fast from their neighbors as the map iterated.
\end{definition}


\begin{theorem} \label{p3-chaos}The existence of periodic-3 orbit alone implies the existence of a large set of sensitive points, or chaotic orbit.
\end{theorem}

We will prove this problem in appendix.

\newpage

%~~~~~~~~~~~~~~~~~~~~~~~~~~~~~~~~~~~~~~~~~~~~~~~~~~~~~~~~~~~~~~~~~~~~~~~~~~~~~~~~~~~~~~~~~~~~~~~~~~~~~













%~~~~~~~~~~~~~~~~~~~~~~~~~~~~~~~~~~~~~~~~~~~~~~~~~~~~~~~~~~~~~~~~~~~~~~~~~~~~~~~~~~~~~~~~~~~~~~~~~~~~~

\section{Two-Dimension and High-Dimension Maps}

\begin{definition}\textbf{Neighborhood}
\\\noindent Consider a $R^n$ space, called every point $x = (x_1, x_2, \ldots x_n)$ is a vector of $R^n$ space,
\\\noindent Define the \textbf{Euclidean Length} $|x| = \sqrt{x_1^2 + x_2^2 + \ldots + x_n^2}$, which is equal to norm;
\\\noindent And define the distance between two point $d(x, y) = |x - y|$;
\\\noindent Also, the \textbf{$\mathbf{\varepsilon}$-neighborhood} is 
$$
\forall \varepsilon > 0, \text{the }\varepsilon \text{-neighborhood of point }p, N_\varepsilon(p) \text{ is } \{x\in R^n | |x - p| < \varepsilon\} \text{, also define }N_\varepsilon^o(p) = N_\varepsilon(p)\backslash\{p\}
$$
\end{definition}

\begin{definition}\textbf{Sink and Source in High-dimension Map}
\\\noindent Let $f$ is a map on $R^n$, $p$ is a vector on $R^n$ which is the fixed point and $f(p) = p$ then 
\\\noindent If there is an $\varepsilon > 0$ s.t. $\forall x \in N_\varepsilon(p), \lim_{k \rightarrow \infty}f^k(x) = p$, then $p$ is a sink or attracting fixed point.
\\\noindent If $\forall x \in N_\varepsilon^o(p), \exists K \text{s.t.} \forall k > K, f^k(x) \notin N_\varepsilon(p)$, then called the point $p$ as source.
\end{definition}
















\end{document}
%~~~~~~~~~~~~~~~~~~~~~~~~~~~~~~~~~~~~~~~~~~~~~~~~~~~~~~~~~~~~~~~~~~~~~~~~~~~~~~~~~~~~~~~~~~~~~~~~~~~~~




















\newpage
\newpage
\appendix
\section{Prove of periodic-3 orbit theory (Theo \ref{p3-chaos})}
Now we try to solve the problem we discussed in periodic-3 orbit of logistic model.


Before we solve this problem, we need some other definition.
\begin{definition}\textbf{Critical point}
\\\noindent Consider a function $f$ in domain $D$, we called every point $x \in D$ s.t. $f'(x) = 0$ or the derivative not exist as \textbf{critical point}.
\end{definition}
\begin{definition}\textbf{Unimodal}
\\\noindent Let $f$ is a map on $R$, if $f$ has and only has one critical point, then we called $f$ is \textbf{unimodal}.
\end{definition}

\begin{example} \label{e-g-p3-orbit}Consider a map in following figure, which is a periodic-3 orbit, denoted $\{A, B, C\}$ s.t. $f(A) = B, f(B) = C, f(C) = A$.\\[4ex]

\begin{figure}[H]
\begin{minipage}[c][0.24\width]{0.24\textwidth}
   \centering
   \includegraphics[width=\textwidth]{figure/section1/period-3-orbit.png}
\end{minipage}
\begin{minipage}[c][0.24\width]{0.24\textwidth}
   \centering
   \includegraphics[width=\textwidth]{figure/section1/period-3-orbit-map.png}
\end{minipage}
\begin{minipage}[c][0.24\width]{0.24\textwidth}
   \centering
   \includegraphics[width=\textwidth]{figure/section1/period-3-orbit-trans.png}
\end{minipage}
\begin{minipage}[c][0.24\width]{0.24\textwidth}
   \centering
   \includegraphics[width=\textwidth]{figure/section1/period-3-orbit-itinerary.png}
\end{minipage}
\\[6ex]\caption{An example periodic-3 orbit map and analysis}\label{period-3-orbit}
\end{figure}

There are some basic conclusion here
\begin{conclusion} \textbf{[i]} $A < B < C, f(A) = B, f(B) = C, f(C) = A$;
\\\noindent \textbf{[ii]} $f$ is unimodal;
\\\noindent \textbf{[iii]} $\forall x \in R \backslash [A, C], f(x) \notin [A, C]$, that means we can take no care of point out of the interval $[A, C]$.
\\\noindent \textbf{[iv]} Let $f([A, B]) = \{y | \exists x \in [A, B] \text{ s.t. } f(x) = y\}$, then $[B, C] \subset f([A, B])$ ({\color{red}\textbf{red}} line and subinterval in second image of Fig. \ref{period-3-orbit})
\\\noindent \textbf{[v]} $[A, B] = f([B, C])$ ({\color{Fuchsia} \textbf{purple}} line and subinterval in second image of Fig. \ref{period-3-orbit})
\end{conclusion}

\end{example}

\newpage 
Now we try to prove the Theo. \ref{p3-chaos} in e.g.\ref{e-g-p3-orbit}.
{\color{blue}
\begin{proof}\textbf{[i]} 

$\blacksquare$
\end{proof}
}


